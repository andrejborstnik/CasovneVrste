\documentclass[12pt,a4paper, notitlepage]{book}
% ukazi za delo s slovenscino -- izberi kodiranje, ki ti ustreza
\usepackage[english]{babel}
%\usepackage[cp12500]{inputenc}
%\usepackage[T1]{fontenc}
\usepackage[utf8]{inputenc}
\usepackage{amsmath,amssymb,amsfonts, amsthm}
\usepackage{bbm}


\usepackage{hyperref}
 
\usepackage{url}
\usepackage{graphicx}
\graphicspath{{./Slike/}}
%\usepackage[normalem]{ulem}
\usepackage[dvipsnames,usenames]{color}

% Knjižnice + ukazi za solution env.
\usepackage{thmtools}

\declaretheoremstyle[
spaceabove=6pt, spacebelow=6pt,
headfont=\normalfont\bfseries,
notefont=\mdseries, notebraces={(}{)},
bodyfont=\normalfont,
postheadspace=1em,
headpunct={},
qed=$\blacktriangleleft$,
numbered=no
]{solstyle}
\declaretheorem[style=solstyle]{solution}
\declaretheorem[style=solstyle]{hint}


% Za risanje diagramov
\usepackage[all]{xy}






\textwidth 15cm
\textheight 24cm
\oddsidemargin.5cm
\evensidemargin.5cm
\topmargin-5mm
\addtolength{\footskip}{10pt}
\pagestyle{plain}
\overfullrule=15pt % oznaci predlogo vrstico
\setlength{\parindent}{0pt}

% Ukazi za barvanje
\definecolor{lightred}{RGB}{255,150,150}
\usepackage{soul}
\usepackage{color}
\newcommand{\hlc}[2][yellow]{ {\sethlcolor{#1} \hl{#2}} }
\newcommand{\hlcr}[1]{\hlc[lightred]{#1}}

% ukazi za matematicna okolja
\theoremstyle{definition} % tekst napisan pokoncnow
\newtheorem{definition}{Definition}[section]
\newtheorem{example}[definition]{Example}
\newtheorem{exercise}[definition]{Exercise}
\newtheorem{observation}[definition]{Observation}
\newtheorem{remark}[definition]{Remark}


\theoremstyle{plain} % tekst napisan posevno
\newtheorem{lemma}[definition]{Lemma}
\newtheorem{theorem}[definition]{Theorem}
\newtheorem{claim}[definition]{Claim}
\newtheorem{proposition}[definition]{Proposition}




% za stevilske mnozice uporabi naslednje simbole
\newcommand{\R}{\mathbb R}
\newcommand{\N}{\mathbb N}
\newcommand{\Z}{\mathbb Z}
\newcommand{\C}{\mathbb C}
\newcommand{\Q}{\mathbb Q}
\newcommand{\F}{\mathbb F}

\newcommand{\Fc}{\mathcal{ F}}
\newcommand{\Gc}{\mathcal{ G}}


% Dodatni operatorji
\DeclareMathOperator{\Var}{Var}
\DeclareMathOperator{\Cov}{Cov}
\DeclareMathOperator{\E}{E}
\DeclareMathOperator{\Sd}{sd}
\DeclareMathOperator{\Span}{span}
\DeclareMathOperator{\Corr}{Corr}
\DeclareMathOperator{\Acf}{acf}
\DeclareMathOperator{\Pacf}{pacf}
\DeclareMathOperator{\Aic}{AIC}
\DeclareMathOperator{\Aicc}{AICC}
\DeclareMathOperator{\Bic}{BIC}
\DeclareMathOperator{\Loglikelihood}{LogLikelihood}


% Operatorji za porazdelitve/modele
\DeclareMathOperator{\Unif}{Unif}
\DeclareMathOperator{\No}{N}
\DeclareMathOperator{\Wn}{WN}
\DeclareMathOperator{\Iid}{IID}
\DeclareMathOperator{\Arma}{ARMA}
\DeclareMathOperator{\Ar}{AR}
\DeclareMathOperator{\Ma}{MA}
\DeclareMathOperator{\Inar}{INAR}
\DeclareMathOperator{\Arima}{ARIMA}
\DeclareMathOperator{\Farima}{FARIMA}
\DeclareMathOperator{\Sarima}{SARIMA}
\DeclareMathOperator{\Garch}{GARCH}
\DeclareMathOperator{\Arch}{ARCH}
\DeclareMathOperator{\Sv}{SV}




% Ukazi za nove in pomembne pojme
\newcommand{\New}[1]{ {\bf \hlcr{#1} } }
\newcommand{\Important}[1]{ {\it \hlc{#1} } }


% Ukazi, ki so zgolj estetskega pomena
\newcommand{\QuestionStart}[1]{ {\it \sc #1} } % Npr. za "Main Question:"



% Naslov
\title{Time Series, Lecture Notes}
\author{prof. Bojan Basrak}
\date{2015/16 }



% Ukaz za footnote brez številk
\makeatletter
\def\blfootnote{\xdef\@thefnmark{}\@footnotetext}
\makeatother

\begin{document}

\maketitle

\blfootnote{ \url{https://github.com/andrejborstnik/CasovneVrste}}   

\newpage



\tableofcontents


\chapter{Basic models and terminology}

Stohastic process $(X_i)_{i \in I}$ on the same probability sapace.

Time series is a sequence of observation, collected over time - prices of stock, level of the sea, ... $ =  (x_i)_{i = 1, \dots, n}$
Ideally at equidestant time points. 

It is also a stohastic process indexed in discrete time, i.e.  $I = \{1, \dots, n\}, \N, \Z$ (model). Typically $I \subset \Z$.
We will be looking at final dimensional distributions of $(X_t)_{t \in \Z}$: family of distributions of a random vector
$(X_{t_1}, \dots, X_{t_k})$, $t_1 < \dots < t_k$.
They determine the distribution of the whole process $(X_t)_{t \in \Z}$.
\[(X_t)_{t \in \Z}: \Omega \rightarrow \R^{\Z} \].

\begin{example}
\begin{itemize}
\item $X_i \overset{i.i.d.}{\sim} F$ is a time series (model). But is independent of time (bad).
\item $X_i \overset{i.i.d.}{\sim} F$, $S_0 = 0$ and $S_n = S_{n - 1} + X_n, n\geq 1$ - random walk. Not interesting, because the model is constantly changing. 
\end{itemize}
\end{example}

\begin{definition}
For a time series $(X_t){t\in \Z}$, satisfying $\Var(X_t) < \infty$ we define  the \New{autocovarinace function} $\gamma_X : \Z \times\Z \rightarrow  \R$
\[ \gamma_X(s,y) = \Cov(X_s, X_t) = \E[(X_s - \E[X_s])(X_t - \E[X_t])] . \]
\end{definition}

\begin{definition}
Time series $(X_t)_t$ is \New{weakly stationary} (by default kind of stationary= weakly stationary) if 
\begin{enumerate}
\item $ \E [|X_t|^2]< \infty$, for all $t$
\item $ \E [X_t] = m$ (constant), for all $t$
\item $ \Cov(X_s, X_t) = \Cov(X_{s+h},X_{t+h})$, $h, s, t \in \Z$
\end{enumerate}
Covariance represents dependance of the variables. 
\end{definition}

\begin{observation}
For weakly stationary time series the function $\gamma_X$ is actually a function of one variable only. 
$\gamma_X : \Z \rightarrow \R$,
\[ \gamma_X(h) = \Cov(X_h, X_0)\].

This is a good definition, because $ \Cov(X_s, X_t) = \Cov(X_{s-t},X_0) = \gamma_X(s-t)$. 
\end{observation}


\begin{definition}
The \New{autocorrelation function} of weakly stationary time series $(X_t)$ is a function
$\gamma_X : \Z \rightarrow [-1, 1]$
\[ \rho_X(h) = \frac{\gamma_X(h)}{\gamma_X(0)} = \Corr(X_{t+h},X_t) . \]
(we assume $\gamma_X(0) > 0$). If it is $0$ everything is a constant and we have no interest in observing it.
\end{definition}

\begin{definition}
Time series $(X_t)$ is \New{strictly (strongly) stationary} if the distribution of a random vector $(X_{t_1 + h}, \dots X_{t_k + h})$ does not depend on $h$ for every $k\in \N, t_1, \dots, t_k \in \Z $.
\end{definition}

\begin{exercise}
Strict stationarity + $ \E [|X|^2] < \infty \Rightarrow$ weak stationarity.
\end{exercise}
\


For a given time series we define a family 
$\{  F_{\bf t} \}_{\bf t \in T}$, where $T = \{(t_1, \dots, t_k), t_1 < \dots < t_k, k \in \N\}$.
$F_{\bf t}({\bf X}) = P(X_t \leq X_1, \dots, X_{t_k \leq x_k}), {\bf x} = (x_1 \dots, x_k), {\bf t} = (t_1, \dots t_k)$
$F_{ \bf t}$ determines the distribution of the time series


\begin{theorem} 
The family $\{F_{\bf t}\}_{\bf t \in T}$ is a class of f.d.d (final dimension distribution) function of a time series
$\iff $
for each $k, {\bf t } \in T$:
\[ \lim_{x_i \rightarrow \infty} F_{\bf t}({\bf x}) = F_{\bf t(i)}  ({\bf x(i)}) , \] 
for every ${\bf x(i)} \in R^{k-1} = (x_1,\dots,x_{k-1},x_{k+1},\dots, x_k)$ and the same for ${\bf t}$ .
\label{th:1}
\end{theorem}
\proof
No proof, very technical. 
$\rightarrow$ is very easy to prove. It is enough to observe the probability is continuous with respect to growing events.
\endproof



\begin{definition}
Random vector ${\bf Y } = (Y_1, \dots, Y_k)^T$ has a multivariate normal (Gaussian) distribution if there exists a 
vector $a \in \R^k$ and a matrix $B \in M_{k\times m}$ and a random vector ${\bf X} = (X_1, \dots, X_m)^T$ such that $X_i 
\overset{i.i.d.}{\sim}  N(0,1)$, such that ${\bf Y} = a + B {\bf X}$.

In this case we know that $\E [ {\bf Y} ] = (\E [Y_1], \dots, \E [Y_k])^T = (a_1, \dots, a_k)$ and the covariance matrix of ${\bf Y}$ is given by $\Sigma_{\bf Y} = B B^T$.
We write $Y \tilde N(a, \Sigma_{\bf Y})$.

\end{definition}

If $ \det (\Sigma_{\bf Y}) > 0$, ${\bf Y}$ has density and 
\[ f_{\bf Y}({\bf y}) = (2 \pi)^{-k/2} \frac{1}{\sqrt{det(\Sigma_{\bf Y} ) } } e^{- \frac{ \langle{\bf y} - a, \sum^{-1}({\bf y} - a)\rangle}{2}}. \]

\begin{definition}
A random process $(X_t)_t$ is G aussian (normal) if its f.d.d.-s are all multivariate normal.
\end{definition} 


\begin{exercise}
Assume $(X_t)_t$ is a gaussian time series with constant expectation. Then the autocovariance function $\gamma_X$ determines all f.d.d-s.
\end{exercise}
\begin{hint}
If we look at the vector $(X_1, \dots)$ we see it is distributed normally. We can read the covariance matrix form $\gamma_X$, and this is all we need to show.
\end{hint}


\begin{exercise}
Assume $(X_t)_t$ is a gaussian time series and weakly stationary. Then it is strictly stationary as well. 
\end{exercise}


\begin{example}
A weakly stationary time series $(X_t)$ is called  \New{ white noise } if $\E [X_t] = 0$ for all $t$ and 
$$\gamma_X(h) =  \begin{cases} \sigma^2 ( = \Var(X_t)) & ,h = 0 \\
			0 & , h \neq 0 \end{cases}
$$
We write $(X_t) \sim WN(0, \sigma^2)$.

Clearly if $(X_t)$ i.i.d. and $ \E [X_t] = 0$ and $ \Var (X_t) < \infty \Rightarrow (X_t) \sim WN(0, \sigma^2)$.  
\end{example}


w w
\begin{exercise}w              aw              w a
Find an example of a strictly stationary white noise, which is not an i.i.d. sequence.
\end{exercise}
\begin{hint} Imagine 
\[ Y \sim \left( \begin{array}{cccc} -2 & -1 & 1 & 2 \\  1/4 &1/4 & 1/4 &1/4 \end{array} \right) , \]
, \[ \epsilon_t  \overset{i.i.d.}{\sim}  \left( \begin{array}{cc} -1& 1 \\ 1/2 &1/2 \end{array} \right) . \]
 \[ X_t = Y \epsilon_t . \]


 We calculate 
w$ \gamma_X$. It is equal to $ \Cov(X_h, X_0) = \E(X_h X_0) = \E(\epsilon_h \epsilon_0 Y^2) = 0$.
\end{hint}

\begin{observation} We will use $(X_t) \sim IID(0, \sigma^2)$ \end{observation}
wd            w    
\begin{exercise}
Find an example of white noise, which is not even strictly stationary.
\end{exercise}

\begin{example} (random walk)
Assume $(X_t) \sim IID(\mu, \sigma^2)$, $X_0$ is independent of $(X_t)_{t \geq 1}$ and $ \Var(X_0) < \infty$.
$ S_0 = X_0$, $S_n = X_0 + \dots + X_n$, then $(S_n)$ cannot be weakly stationary, unless $ \mu = \sigma = 0 $.
Clearly if $\mu \neq 0$ then $ \E[S_n] = \E[X_0] + n\mu$ is changing!
If $\mu = 0$ and $\sigma^2 > 0$, then 
\[ \gamma_S(h,h) = \Cov(S_h, S_h) = \Var(S_h) = \Var(X_0) + n \sigma^2, \]
 which is growing. So we cannot have a stationary random walk.
\end{example}

\begin{observation}
Definition of stationarity (weak and strong) and functions $\gamma_X, \rho_X$ can be extended to one-sided time series $(X_t)_ {t\geq 0}$. 
\end{observation}


\begin{example}
Assume $A_1, A_2$ are uncorrelated random variables with expectation $0$ and equal variance $\sigma^2$. Then the time series 
\[ X_t = A_1 \cos{\nu t} +A_2 \sin{\nu t} \] . $\nu > 0$.
is $2 \pi / \nu$ periodic and weakly stationary.
Obviously $ \E[X_t] = 0$ and $\E[X_t^2] < \infty$.
\begin{align*} \Cov(X_{t+h}, X_t)  = \E[ A_1^2 \cos{\nu t}\cos{t + h} + A_2^2 \sin{\nu(t + h)}\sin{\nu t} ] \\
= \sigma^2 \cos(\nu (t + h) - \nu t) = \sigma^2 \cos(\nu h). \end{align*}
There is no $t$ here. So $X_t$ is stationary. 

Just by looking at the sample we cannot know if it is randomly generated. We also do not know if it is stationary (it is $\sin$).
But it is. So it is difficult to analyze in practice.
\end{example}


\begin{exercise}
Assume $A, \theta$ are independent random variables. $\Var(A) < \infty$ and $\theta \sim \Unif[0,2\pi]$, $\nu > 0$ then the time series
\[ X_t = A \cos{( \nu t + \theta )} \] 
 is weakly stationary. 
\end{exercise}
\begin{hint}
You write the $\cos$ using the trigonometric formula and notice it is the same as in the previous example for 
$A_1 = A\cos{\theta}$ and $A_2 = A\sin{ - \theta}$.
\end{hint}

\begin{example} (MA(1))
Assume $(Z_t) \sim WN(0, \sigma^2), \theta \in \R$
and 
\[ X_t = Z_t + \theta Z_{t-1},  t \in \Z. \]
Process $(X_t)$ is called \New{ moving average of order one} $= MA(1)$ .

Its expectation is clearly $0$. It is also a weakly stationary time series (strongly if $Z_t$ is strongly stationary):
\begin{align*} \gamma_X(h) =  \Cov(X_{t + h}, X_t) =  \Cov(Z_{t+h} + \theta Z_{t+h-1}, Z_t + \theta Z_{t-1})  \\ = 
= \begin{cases} \sigma^2(1 + \theta^2)  &, h = 0 \\
	\theta \sigma^2	& , h = +-1 \\
		0	& , otherwise \end{cases}
\end{align*}
This clearly does not depend on $t$, so $(X_t)$ is weakly stationary. We also observe that $\gamma_X(h) = 0$ for all $|h| \geq 2$. We say that $(X_t)$ has \Important{\ short memory}.

\end{example}

\begin{observation}
Time series theory makes a distinction between:
\begin{itemize}
\item short range dependent models (like MA(1))
\item long range dependent models
\end{itemize}
\end{observation}


\begin{example} ($\Ar(1)$)
Assume $(Z_t) \sim WN(0, \sigma^2)$, $\phi \in \R$ and that $(X_t)$ satisfies 
\[  X_t = \phi X_{t-1} +Z_t, t \in \Z .  \label{eq:1} .\] 


For different values of $X_0$ we can generate different values of $(X_t)$ process. The main question is, if we can (in this setting) find a stationary times series $(X_t)$, which satisfies
the equation above.

It turns out we cannot always do that. It will depend on $\phi$. $\phi = 1$ gives us a random walk and it is not stationary for non-stationary $(Z_t)$.

If a stationary solution exists it is unique.

Assume $|\phi| < 1$. We can guess the solution by iterating one step backwards. We get:
\[ X_t = \phi^k X_{t-k} + \phi^{k-1}Z_{t-k} + \dots + Z_t  . \] 
One can guess that the stationary solution   is $X_t = \sum_{k = 0}^\infty \phi^k Z_{t-k}$, which is a random series. 

Does this series converge? What does this mean for random variables? (which type of convergence?) 

We will show the series is converging a.s. (even absolutely). That will mean $X_t$ is well defined for almost all $\omega$-s. Then it satisfies  
\ref{eq:1} (check directly) and it is weakly stationary. 

\[ \E[X_t] = \E\left[\left(\sum \phi^k Z_{t-k} \right) \right] = \sum \phi^k \E[Z_{t-k}] = 0 . \] 
We would like to put the expectation under the sign (but this is not true necessarily). 

For $h \geq 0$: 
\begin{align*} \gamma_X(h) = \Cov(X_{t +h }, X_t) = \E \left[X_{t+h}X_t \right] \\ 
= \E \left[ \sum_i \sum_j \phi^i \phi^j Z_{t + h - j} Z_{t + h - i} \right] \\ 
\overset{interchanging}{=}   \sum_i \sum _j \phi^i \phi^j \gamma_Z(h-i+j). \end{align*}
(This gives $\sigma^2$ only when $h+j = i$.)
\begin{align*} = \sum_j \phi^j \phi^{h + j}\sigma^2 = \sigma^2 \phi^h \frac{1}{1 - \phi^2}. \end{align*}

By symmetry we get (for $h \leq 0$):
\[ \gamma_X(h) = \sigma^2 \frac{\phi^{|h|}}{1 - \phi^2} . \]

This is decreasing exponentially fast! We used Lebesgue dominated convergence.


When $\phi = 1$ we cannot find a stationary solution at all:
\[X_t = X_{t-1} + Z_t = \dots = X_0 + Z_1 + \dots + Z_t. \]
 Meaning $(X_t)$ is a random walk. 

Assume $\Var(X_0) < \infty$. $\Var(X_t - X_0) = t \sigma^2$. 
Standard deviation $ \Sd(X_t - X_0) = \sqrt{t} \sigma$.

(We will show that $|| X || = \sqrt{ \E[X^2]}$ is a norm ).
\begin{align*} 
= \sqrt{\E \left[ (X_t - X_0)^2\right]  - \left(\E[X_t] - \E[X_0]\right)^2} \\ 
= || X_t - X_0 - (\E[X_t] - \E[X_0])^2 ||  
\\ \leq || X_t - \E [X_t]  || + || X_0 - \E[X_0]|| 
\\ =  \Sd(X_t) + \Sd(X_0) \in [0, \infty] . \end{align*}

So when $\sqrt{t}\sigma \rightarrow \infty$ the s.d. also goes towards $\infty$. Therefore $(X_t)$ cannot be weakly stationary (for $\sigma \neq 0$). 
\end{example}



\begin{exercise}
Show that there is no stationary solution of 
\ref{eq:1} 
 if $\phi = -1$.
\end{exercise}

\begin{exercise}
Show that the stationary solution of 
\ref{eq:1}  for $|\phi| > 1 $ is of the form 
$X_t = \sum_{k = 1}^\infty (1/\phi)^k Z_{t + k}$.
\end{exercise}

\begin{example}(ARCH(1))
Assume $(S_t)$ are prices of financial assets. Log-returns are 
\[X_t = \log{S_t / S_{t-1}} = \log{1 + \frac{S_t - S_{t-1}}{S_{t-1}}}. \]
 $ X' = \frac{S_t - S_{t-1}}{S_{t-1}} $ is relative return. 
$ X' \approx X_t'$. So it measures how much money you made (approximately).

$\Ar(1)$ = auto-regressive model of order one:

Typically people assume that 
\[ X_t = \sigma Z:t ,  Z_t  \overset{i.i.d.}{\sim} N(0,1). \]
 \Important{Black-Scholes model}. It is a benchmark model, because it is simple. But it is not correct (realistic).



Engel (1982):
Stylized facts contradict i.i.d. assumption. The normal distribution has small tails. Stock prices go up and down -> clustering.


\begin{align}
\begin{cases}
Z_t \overset{I.I.D}{\sim}  N(0,1) & , ( \text{or more generally} Z_t \sim IID(0,1) ) \\
X_t = \sigma_t Z_t & , \\
\sigma_t^2 = \alpha_0 + \alpha_1 X_{t-1}^2 &, \alpha_0, \alpha_1 \text{constant}. 
\end{cases}
\label{eq:2}
\end{align}

$\sigma_t^2$ is sometimes called volatility.
 

ARCH = auto-regressive conditionally heteroscedastic (heteroscedastic = variance is changing)

\QuestionStart{Main question:}  \\ Does 
\ref{eq:2} have a stationary solution? 

For $\alpha_1 \in (0,1)$ one can find a stationary solution with the same trick as before (by iterating backwards).

\[ \sigma_t^2 = \alpha_0 + \alpha_0 \sum_{j = 1}^\infty \alpha_1 Z_{t-1}^2 \cdots \alpha_1Z_{t-j}^2 
\label{eq:3} \] 



This series is converging almost surely (a.s.). 
Series in 
\ref{eq:3} is only made of non-negative terms. 

If we show that the expectation of this formula is finite, then we know that the formula is converging a.s..
\[ \E[ \sigma_t^2] = \E\left[\alpha_0 + \alpha_0 \sum_j \alpha_1^j Z_{t-1}^2 \cdots Z_{t-j}^2 \right]  = (*). \]
Because all terms are non-negative we can exchange $ \E$ and  $\lim$ because of Fubinni (for integrals).
\[ (*) = \alpha_0 + \alpha_0 \sum_j \alpha_1^j = \alpha_0 \frac{1}{1-\alpha_1} < \infty. \]
 Therefore $\sigma_t^2 < \infty$ a.s.

The pair $(\sigma_t^2, \sigma_t Z_t = X_t)$ is well defined and solves the  
\ref{eq:2}.

Observe $\sigma_t \in \sigma\{ Z_{t-1},Z_{t-2}, \dots\}$. So $\sigma_t$ is independent of $Z_t, Z_{t+1} \dots$.
\[ \E[X_t] = \E\left[ \E\left[ X_t | \sigma\{ Z_{t-1}, \dots\} \right] \right]  = (*). \]
 We can take $\sigma_t$ out, if it is a constant (and it is, because we know
 the sigma algebra and therefore all values before $t$. In other words $\sigma_t$ is measurable in the sigma algebra). 
\[ (*) = \E \left[ \E \left[ Z_t \right] \sigma_t  \right] = \E \left[ \sigma_t \cdot 0 \right] = 0 . \]

We can do the same for 
\begin{align*} \E\left[ X_t^2 | \sigma\{\dots\} \right]  =  \E \left[  \sigma_t^2 Z_t ^2 | \sigma \{ Z_{t-1}, Z_{t-2} \ldots \} \right] \end{align*}
\begin{align*} \Var(X_t) = \E[X_t^2] = \E[\E[ \sigma_t^2 Z_t^2 | \sigma\{\dots\} ] ] \\
 = E(\sigma_t^2) = \frac{\alpha_0}{1 - \alpha_1}. \end{align*}

So variance does not change. Only the conditional variance does.

Similarly one can show that $X_t$ -s are uncorrelated. So $ \Cov(X_{t+ h}, X_t) = 0, h \neq 0$. It follows that $(X_t)'$ are white noise. This does not contradict the reality. Prices are not correlated, but are dependent.( Because the sign is unpredictable.)


It was later generalized by Boleslev (1986) into \New{GARCH} (generalized ARCH). it is the most often used model, that models real-life data.

In our example $X_t$ is also strictly stationary (it is weakly, because it is white noise):
Since $X_t = \Phi(Z_t, Z_{t-1}, Z_{t-2}  \dots) ) ,  Z_t ' $ -s iid. (idea of proof). 
\end{example}

\section{Properties of $\gamma$ and $\rho$}

\[\gamma_X(h) = \Cov(X_{t+h}, X_t), h \in \Z \]
\[ \rho_X(h) = \Corr(X_{t+h},X_t) = \frac{ \gamma_X(th) }{ \gamma_X(0) } \]

\begin{lemma}
\begin{enumerate}

\item $\gamma_X(0) \geq 0$
\item $|\gamma_X(h)| \leq \gamma_X(0)$ .
\item $\gamma$ is an even function .
\item $\gamma$ is a positive semidefinite function i.e. for all $n$ and for all $(a_1, \dots, a_n) \in \R^n$ 
\[ \sum_{i=1}^\infty \left( \sum_{j=1 }^\infty \left( a_i \gamma(i -j) a_j \right) \right) \geq 0 . \]
\end{enumerate}
\end{lemma}

\proof
\begin{enumerate}
\item Trivial (property of $ \Cov $ ).
\item Schwartz-Cauchy inequality (covariance can be seen as inner product).
\item Trivial ( weak stationarity + symertry of $ \Cov $).
\item the sum equals $Var(a_1X_1 + \dots + a_n X_n) =$
\[  \sum_{i,j} \Cov(a_iX_i,a_jX_j) = \sum _{ i,j} a_i a_j \gamma(i-j) . \] 
\end{enumerate}
\endproof

\begin{observation}
The above sum i equivalent to the fact, that the
matrix 
\[ \Gamma_n = 
\left( \begin{array} {c c c c} 
 \gamma(0) & \gamma(1)  & \ldots & \gamma(n-1) \\
\gamma(1) & \gamma(0) & \gamma(1) \ldots & \gamma(n-2) \\
\vdots & \ddots & \ddots & \vdots \\
\gamma(n-1) & \ldots & \gamma(1) & \gamma(0)
\end{array} \right)
\]

is positive semidefinite for all $ n \in \N $ .
\end{observation}


\begin{theorem}
Function $\gamma: \Z \rightarrow \R$ is auto-covariant function of some weakly stationary time series 
$ \iff  $ f
$\gamma$ is even and positive semidefinite.
\end{theorem}


\proof
($\Rightarrow$) proved in lemma.

($ \Leftarrow$) Because of positive semidefinitness of $\Gamma_n$ for all $n$ it follows that there exists $N(0, \Gamma_n)$. $
\Gamma_n$ can be a covariance matrix. These distributions as a family of f.d.d-s are consistent. Meaning they satisfy the 
condition of the theorem 
\ref{th:1}.

Theorem \ref{th:1} now implies that we can find a (gaussian) sequence that has $\Gamma_n$ as its autocovariance function.

\endproof

\begin{observation}
Function $\rho$ has the same properties as function $\gamma$. We also know that $\rho(0) = 1$ (always).
\end{observation}

\begin{observation}
It is difficult to test for positive semidefinitness. Sometimes it is easier to construct a sequence directly.
\end{observation}

\begin{exercise}
Show that the following functions are positive semidefinite.
\begin{itemize}
\item \[ \gamma(h) = 2 . \]
\item \[ \gamma(h) = \begin{cases} 2 & , \text{if } h = 0 \\
  1 & , \text{if } h = +-1 \\
 0 &  ,  \text{otherwise } . \end{cases} \]
\item \[ \gamma(h) = a \cos{\nu h}, a, \nu > 0 . \]
\end{itemize}
\begin{itemize}
\item  \[ X_t = X \sim N(0,2) , \]
\item is the autocovariance of MA model (done before)
\item $\gamma$ is the autocorr. of periodic seq. $X_t = \dots$ (done before)
\end{itemize}


\end{exercise}

\begin{exercise}
If $\gamma_1$ and $\gamma_2$ are even and positive semidefinite functions the same holds for the function 
$\gamma = \alpha_1 \gamma_1 + \alpha_2 \gamma_2, \alpha_1, \alpha_2 \geq 0$. 

% Kaj je to? V zvezku tega ni!
%We sum the two series, multiplied not by $\alpha_1$, but some function of it.

In practice we need to estimate $\gamma_X, \rho_X$ and $\mu = \E[X_t]$. Natural estimators are easy to find:
\[ \overline{X} _n = \frac{1}{n} \left(X_1 + \dots + X_n \right) \]  the arithmetic mean

For  $X_i $ i.i.d. it converges (law of large numbers). Even a stronger results is the central limit theorem.

We do not have i.i.d. $ X_ i $ -s, because we have dependance. We can do:
\[ \hat{\gamma(h)} =  \frac{1}{n} \sum_{i = 1}^{n- h}   (X_i - \overline{X}_n)(X_{i+h} - \overline{X}_n) , h \in \{ -(n-1), \ldots -1, 0, 1, \ldots n-1 \} . \]
Obviously $h < n$ must hold. We say $h =$ {\bf  lag}.
We divide by $n$ because such a matrix is again positive definite. But it is not unbiased (we would have to divide 
by the $n - 1 - $ number of missing terms $= h$). In this case we would not have a positive definite matrix. 

For $h < 0$ we set $\gamma(h) = \gamma(-h)$.

$\hat{\rho(h)} = \hat{\gamma(h)} / \hat{\gamma(0)}$.

\end{exercise}





\section*{First steps of practical analysis of time series}

% RKoda


R libraries:

tseries

TSA

stats

evir

fGarch $->$ not perfectly written ?

mean

sd

read.table

ts.plot $->$ on the same graph (for multivariable)

plot.ts $->$ on more graphs

acf   plots $\hat{\rho}$

acf also works for squares. Our job is to explain this $->$ we are outside the boundaries for siemens data form R.

arima.sim
























\newpage

\section{Basic transformations of time series}

\vskip1cm

The goal of each transformation is to make time series stationary (or to appear to be stationary).
Why is stationarity important? If the distribution is changing it is hard to predict anything.


\begin{example} (\New{ log/relative returns})
$ S_t  \ldots $ financial data
\[ (S_t) \rightarrow (X_t), X_t = \log{S_t/S_{t-1}}  \] 
\[  \rightarrow (X_t'), X_t' = \frac{S_t - S_{t-1}}{S_{t-1}} \]
\end{example}

We define the \New{ backwards shift operator} 
\[ B(X)_t = (BX)_t = X_{t-1}, t \in \Z . \]

Notice that it maps $(X_t)_t$ into $(X_{t-1})_t$.

% on piše grad, govori pa delta...
We also define the \New{ differencing operator} 
\[ \nabla(X)_t = \nabla X_t = (I - B) X_t = X_t - X_{t-1} . \]
 In other words
\[ \nabla = I - B . \]

Recursively for $j \geq 1$ we define 
\[ B^j X_t = B (B^{j-1} X_t) , \]
 and the same for $\nabla$.

Note that $\nabla^0 = B^0 = I$. 

\begin{example}
Assume $(Y_t)$ is strictly stationary and $a,b \in \R, $
\[ X_t = a t + b + Y_t, \]
 then $X_t$ cannot be stationary unless $a = 0$. (homework ... use characteristic functions)

But 
$\nabla X_t = at + b + Y_t - a(t-1) - b - Y_{t-1} = a + Y_t - Y_{t-1}$ is stationary ($Y_t$ and $Y_{t-1}$ have the same distribution).  
\end{example}


\begin{exercise}
Assume $(Y_t)$ is weakly stationary in the previous example. Show that $(X_t)$ is weakly stationary as well.

\end{exercise}
\begin{hint} Use the lemma below.
\end{hint}

\begin{exercise}
Show that
\[ \nabla^2 X_t = X_t - 2X_{t-1} + X_{t-2} \] 
 for all $t$ and use that to determine a general form of $\nabla^k, k \geq 1$.
\end{exercise}


\begin{exercise}
Assume $(Y_t)$ is weakly (strongly) stationary. Assume 
\[ n_j = \sum_{j=0}^k a_j t^j , \]
 and 
\[ X_t = n_t + Y_t . \]

Show that 
\[ \nabla^k X_t = k! a_k + \nabla^k Y_t , \]
 which is again weakly (strongly) stationary.  
\end{exercise}

\begin{observation}
Differencing can remove any polynomial trend. 
\end{observation}


Assume that time series $(Y_t)$ satisfies 
\[ Y_t = \sum_{k = -\infty}^\infty h_k X_{t-k}, t \in \Z , \]
 (assume it converges) for some sequence $h_k$ and some time series $(Y_t)$. Than we say that $(Y_t)$ is a product of action of \New{ liner filter} on the sequence $(X_t)$.  

Formally we denote  the \New{filter } by 
\[\psi(B) =  \sum_{k = -\infty}^\infty h_k B^k . \]

\begin{example}
$\nabla = B^0 - B^1$, so $h_0 = 1, h_1 = -1$ and $0$ otherwise.

\end{example}

If $(h_k)$ is absolutely summable and $(X_t)$ is a time series, which satisfies $X_t = \sum h_k Z_{t-k}$ for $Z_t$ white noise we say $(X_t)$ is a \New{ liner process} , notation:
\[ X_t = \psi(B) Z_t . \]

So linear process is an absolutely summable linear filter acting on white noise. 



\begin{example}(MA(1))
\[ X_t = Z_t + \theta  Z_{t-1} ,  \quad Z_t \sim WN(0, \sigma ^2) . \]
\end{example}


The condition of absolute summability ensures $X_t$ is well defined (the sum converges).

\begin{lemma}  \Important{( Key lemma on linear transformations) }
Assume $(Z_t)_t$ is a time series and the sequence $(h_k)_k$ satisfies the absolute summability condition. Then 
\begin{enumerate}
\item $ \sup_t{\E[|Z-t|] } < \infty \Rightarrow  \sum_k h_k Z_{t-k}$ converges absolutely a.s. and in $L_1$. 
\item $ \sup_t{\E [Z_t^2]} < \infty \Rightarrow \sum_k h_k Z_{t-k}$ converges in $L_2$ as well.
\item $Z_t$ weakly stationary $\Rightarrow$ the series $X_t = \sum_k h_k Z_{t-k}$ is weakly stationary as well. Moreover \\\[ \gamma_X(k) = \sum_{i,j} h_i h_{j + i - k} \gamma_Z(i) . \]
\label{lem:key}
\end{enumerate}

\end{lemma}


Reminder:
$Y_n \rightarrow (L_p) Y$ means $E|Y_n - Y|^p \rightarrow 0$. 


\begin{exercise}
Assume $(X_t)$ satisfies $X_t \geq 0$ for all $t$. Then $E\sum_t X_t = \sum_t E X_t$ (can be $+ \infty$).

\end{exercise}
\begin{hint} Lebesgue theorem on monotone convergence)
 \[ \sum_{|t| \leq m} \rightarrow \sum_t X_t . \]
\end{hint}


\begin{exercise}
Assume $\sum_t E|X_t| < \infty$. Then $\sum_t X_t$ converges a.s. and in $L_1$. Moreover $E\sum_t X_t = \sum_t E X_t$.
\end{exercise}
\begin{hint}
Again lebesgue, but on dominant convergence)
\end{hint}

\proof (of lemma)
\begin{enumerate}
\item Observe 
\[ \sum_k \E[|h_k Z_{t-k}|] \leq \sup_t \E[|Z_t|] \sum_k |h_k|. \]
 Now the previous exercise proves the series converges (everything $< \infty$). 

\item 
\[ \sup_t \E[ Z_t^2 ] < \infty \Rightarrow \sup \E [ | Z_t | ] < \infty \]
This is consequence of the Cauchy Schwarz  also $ |z| \leq 1 + z^2. $ 
Now the series is converging, according to (1). 

Consider the 

\begin{align*} 
\E\left[\left(\sum_j h_j Z_{t-j} - \sum_{|j|\leq k} h_j Z_{t-j}\right)^2 \right] = \\
\E \left[ \left( \sum_{|j| > k} h_j Z_{t-j} \right)^2 \right]  \leq  \\ 
\E\left[ \left(\sum_{|j| > k} |h_j| |Z_{t-j}| \right)^2  \right] = \\
 \sum_{ |i|,|j| > k} |h_i||h_j|  |\E[ |Z_{t-i}||Z_{t-j}  ] |\leq \\
\sqrt{ \E [ | Z_{t-i} |]^2 \E [ | Z_{t-j} | ]^2 }   \leq \\ 
\sup _t \E [ Z_t]^2 \leq 
\sup _t \E [Z_t] ^2 \sum _{|i|, |j|>k} |h_i | |h_| = \\
 \sup _t \E [ Z_t ] ^2 \left( \sum _{|i|>k} |h_i|^2 \right)  \rightarrow 0
\end{align*}
\item
By (1) $ X_t $ is well defined and converges a.s. in $L_1 $. If we use exercise 15, we get:

\begin{align*}
\E \left[ \sum _{j \in \Z} h_j Z_{t-j} \right] = \sum _{j \in \Z} \E[ Z_{t-j} ] \end{align*}
And since $ Z_t $ is weakly stationary 
\[  \E [ Z_t] = \mu _Z . . \]
Without loss of generality we assume $ \mu _Z = 0 $ . 
\begin{align*} \Cov(X_{t+h} , X_t) = \E [ X_{t+h}, X_t ] = \\
\E\left[ \sum_i \sum _j h_i Z_{t+h-i} h_j Z_{t-j} \right] = (*) 
\end{align*}
We define $ l = h-i+j $ , again use Cauchy Schwarz inequality 

\begin{align*}
(*) =  \sum_i \sum_l  h_i h_{i+l-h} \gamma _Z(l) = \\
\gamma _X(h) . \end{align*}
Since the last equality does not depend on $ t $ the time series $ (X_t) $ is weakly stationary.

\end{enumerate}
\endproof


\begin{exercise}
Say $Z_t \sim WN(0, \sigma^2 )$. For convergence of series $\sum_k h_k Z_{t-k}$ it is sufficient that $\sum h_j^2 < \infty$.
\end{exercise}
\begin{hint}
 Use Hilbert space theory to prove it)
\end{hint}

Linear filter $\psi(B) = \sum_k h_k B^k$ is \New{ causal} if $h_k = 0$ for $k < 0$. 

A linear process 
\[ X_t = \sum:k h_k Z_{t-k}\]  is \New{ causal} (with respect to white noise $(Z_k)$) if $h_k = 0$ for $k < 0$. 

\begin{example}
For $|\phi| < 1 , (Z_t) \sim WN(0,1)$ the series 
\[ X_t = \sum_{j=0}^\infty \phi^j Z_{t-j} , \]
 is well defined and solves the $\Ar(1)$ equation. Moreover $(X_t)$ is causal. 
\end{example}

Assume 
\[ \psi(B) = \sum_j \psi_j B^j, \phi(B) = \sum_j \phi_j B^j , \]
 and $\phi_j, \psi_j$ are absolutely summable. For the time series $(Z_t)$ such that $\E[|Z_t|] < \infty$ define $X_t = \psi(B) \phi(B) Z_t$.

Use the lemma 
\ref{lem:key}
to see everything is well defined. 

\[X_t = \sum_k \psi_k \sum_j \phi_j Z_{z-k-j} = \sum_{k,j} \psi_k \phi_j Z_{t-k-j}  = (*) . \]

Set $k+j i$, hence $ j = l-k$.

\[ (*) = \sum_l Z_{t+l} \sum_k \psi_k \phi_{l-k} = (*) \]

We define $ \theta_l = \sum_k \psi_k \phi_{l-k}$ and get

\[ (*) = \Psi(B) \circ  \Phi(B) = \sum_l \theta_l B^l . \]

It follows that 
\[\Psi(B) \circ \Phi(B) = \Phi(B)  \circ \Psi(B) . \]


\newpage

\section{On practical time series analysis}
\vskip1cm


If we cannot assume stationarity we will assume that we study time series of the following form:

\[X_t = m_t + S_t + Y_t , \]
 where
\ the \New{ trend} $m_t$ is a deterministic (hence the small letter) sequence, which "does not vary too much",

the \New{ seasonality} $s_t$ a deterministic sequence of known period $d$, such that 
\[ s_t = s_{t+d} ,  \text{ for all } t , \]

and the $Y_t$ a (weakly/strongly) stationary time series.


\begin{observation}
In practice it is hard to distinguish the three components. 

\Important{Global warming} : trend or seasonality? Here it is proven, but it can be hard to determine.
\end{observation}

But how do you do it, if you have to?

Practical time series analysis (t.s.a.) can be split into several steps:
\begin{enumerate}
\item plot the observations $x_t$. We want to detect possible trend and seasonality. 
\item try to remove the trend and seasonality
\item residuals $y_t = x_t - \hat{m}_t - \hat{s}_t$ or $y_t = \log{x_t / x_{t-1}}$ are then modeled  by a stationary model
\begin{enumerate}
\item choose a model ($\Ar, \Ma, \Arch , \dots \Arma$)
\item estimate parameters
\item check goodness of fit
\end{enumerate}
\end{enumerate}
After that we can try to predict or simulate future behavior of the time series.

Use Ockham's razor.

Box (person): All models are wrong. Some are useful.

Einstein: Make the model as simple as possible, but not simpler than that.

\section{ Decomposing time series}

\vskip1cm


Assume 
\[ X_t = m_t + s_t + Y_t, \]
 $Y_t$ a weakly stationary sequence, $s_t = s_{t+d}$ and $\sum_{t=1}^d s_t = 0$ (otherwise we can subtract the mean in $m_t$) and $ \E [Y_t] = 0$ (similar as before). 

Assume (for start) there is no seasonality. 
\begin{enumerate}
\item We can remove the trend by differencing. We have seen that if $m_t = \sum_{j=0}^ka_j t^j$ is polynomial and we apply $\nabla^k$ we get
$X_t = k! a_k \nabla^k Y_t$. Because $Y_t$ is stationary $X_t$ is stationary by lemma
\ref{lem:key} .
.
\item Direct estimation of the trend. 
\begin{enumerate}
\item smoothing. Take $q < n \in \N$ ("not too small"). We can set $\hat{m}_t = 1 / (2q + 1)\sum_{j = -q}^q X_{t+j}$, which is a linear filter. 

\[ \hat{m}_t = \frac{1}{2q + 1}\sum_{j = -q}^q m_{t+j}+ 1 / (2q + 1)\sum_{j = -q}^q Y_{t+j}. \]

\[ \frac{1 }{ 2q + 1}\sum_{j = -q}^q Y_{t+j} = (\approx) 0 \]  by law of large numbers??

If $m_t$ does not vary too much it is piecewise linear and we can model it this way and it will be approximately true.

But this approach does not work on the edge of the time series. 

\item exponential smoothing. Take $\alpha \in (0,1)$. $\hat{m}_1 = X_1$. 
\begin{align*}  \hat{m}_t = \alpha X_t + (1 - \alpha) m_{t-1} = \\
\sum_{j = 0}^{k-2} \alpha(1-\alpha) X_{t-j} + (1-\alpha)^{t-1} X_j.
\end{align*}


\item fitting a function to $m_t$, e.g. polynomial

\begin{align*}  \hat{m}_t  = & \sum_{j = 0}^k \hat{a}_jt^j   \\
 \min _{ \hat{a}_j } \sum _{j=0} ^k \hat{a} _j t^j \end{align*}
and we choose $\hat{a}_k $ by OLS (ordinary least squares).

\end{enumerate}
 \item Remove seasonality and the trend by differencing. Define operator 
\[ \nabla_d = X_t = X_t - X_{t -d} = (I - B^d)X_t. \]
If $s_t = s_{t+d}$ then 
\[ \nabla_d X_t = m_t - m_{t-d} + s_t - s_{t-d} + Y_t - Y_{t-d}, \]
 but the $ s_t - s_{t-d} = 0$, so we come back to the previous examples, it works only for $ t \in \{q +1, \ldots n-1 \} $.

\item Direct estimation of trend and seasonality (can be done in many ways, we show one).
\begin{enumerate}
\item Set 
\[ q = \begin{cases} \frac{d}{2} & \text{, if } d \text{ even,}  \\ \frac{d-1}{ 2} \text{, if } d  \text{ is odd. } \end{cases} \]
 The initial estimate of the trend is chosen by smoothing:

\[ d = 2q, \]
\[ \hat{w}_t = \frac{1}{2d}X_{t-q} + \frac{1}{d} X_{t-q+1} + \frac{1}{d}X_{t+q-1} + \frac{1}{2d}X_{t+q}. \]

Here we are trying to estimate the trend. Say we have monthly data and $q = 12$. We estimate by averaging over one year. But the edge months would be counted twice (previous year, next year), so we only take half of that. 

\item Transform data by removing the initial estimation of the trend: 
\[ X_t \rightarrow X_t - \hat{w}_t. \]
Now we also estimate the seasonality. Initially we set 
\[ \hat{u}_k = \frac{1}{ \lfloor\frac{n-k}{d}\rfloor + 1 } \sum_{j=0}^{\lfloor\frac{n-k}{d}\rfloor} (x_{j+k} - \hat{w}_{j+d+k}), k=1,\dots, d \]

\[ \hat{s}_k = \hat{u}_k - 1/d \sum_{j=1}^d \hat{u}_j, k=1, \dots, d, \hat{s}_t = \hat{s}_{t-d} . \]

\item we transform 
\[ X_t \rightarrow X_t - \hat{s}_t = d_t. \]
 For $d_t$ we estimate trend by one of the methods above. 

\end{enumerate}  
\end{enumerate}


R has many (special) procedures to compute these estimates (use as a black box). 


\vskip0.5cm

Again R on screen:

hist

qqnorm (qqplot package)

filter

HoltWinters

lagged plot (plot $(X_t, X_{t+d})$). If no pattern, there is no dependance

decompose

stl






















\newpage
\chapter{Hilbert spaces, Conditional expectation, and Prediction}

\vskip1cm

Problem: I have given observations $X_1, \dots, X_n$. The goal is to predict $X_{n+1}$. 

The traditional way would be to minimize RMSE 
\[ \min_g \E[ (X_{n+1} - g(X_1, \dots, X_n))^2 ] . \]

A simpler problem would be to find $ \E[(X_{n+1} - (\alpha_1 X_n + \dots + \alpha_n X_1))^2 ] $. ($X_n$ most important, hence $\alpha_1$)

\begin{definition}
\New{ Hilbert space} is a complete inner product space.
\end{definition}

\begin{theorem} \New{ (Projection theorem) }
If $M$ is a closed subspace of a Hilbert space $H$ and $x \in H$, then
\begin{enumerate}
\item there is an unique $\hat{x} \in M$, such that
\[ || x - \hat{x}|| = \inf_{y \in M}|| x - y || . \]
\item $\hat{x} \in M $ 
\[ || x - \hat{x}|| = \inf_{y \in M}|| x - y || \] is equivalentHilbert to $x' \in M$, 
\[ x - \hat{x} \perp M , \] i.e.
\[ (x - \hat{x}) \in M^{\perp} \]
\end{enumerate}
\end{theorem}
\proof
No proof. Simple.
\endproof


The mapping $x \rightarrow \hat{x}$ is denoted by $\Pi_M$. Clearly it is a linear operator, $|| \Pi_M X|| \leq ||x||$ for all $x \in H$ and $\Pi_M^2 = \Pi_M$. 

For $M_1 < M_2$ (closed subspace) we know 
\[\Pi_{M_1}\Pi_{M_2}x = \Pi_{M_1}x. \]

If $M_1 \perp M_2$ closed subspaces of $H$ then 
 \[ \Pi_{M_1 + M_2} x = \Pi_{M_1}x + \Pi_{M_2}x. \]

We will only be interested in the Hilbert space $L_2(\Omega, F, P)$. We say $X \in L_2(\Omega, F, P)$ if $\E[X^2] < \infty$.

Although actually $[X] \in L_2(\Omega, F, P)$, we say 
 (equivalent classes are contained in $L_2$). 
\[ X \sim Y \iff P(x \neq Y) = 0. \]
\begin{observation}
One can consider complex valued random variables as well. The inner product $X, Y \in L_2(\Omega, F, P)$
\[  \langle X,Y \rangle = E(X \overline{Y}). \]
\end{observation}

Thus for real valued $X,Y$ conjugation is not needed. 
\[ ||  X || = ||\sqrt{\langle X,X \rangle}|| = \sqrt{EX^2} \]
Convergence in $L_2$ is equivalent to convergence in this norm.

Cauchy-Schwartz:
\[ |\langle X,Y\rangle |^2 \leq ||X||^2 ||Y||^2 \label{eq3:1} \]
i.e.
\[ |\E[ | X Y| ] \leq \sqrt{\E[X^2]  \E[Y^2]}. \]

\begin{example}
If $X_n \overset{L_2}{\rightarrow} X, Y_n \overset{L_2}{\rightarrow} Y$ then
\[  \langle X_n, Y_n\rangle \overset{n \to \infty}{ \rightarrow} \langle X,Y \rangle  . \]
\end{example}
\proof
We observe the difference 
\begin{align*} |\langle X_n, Y_n\rangle  - \langle X,Y\rangle | \leq |\langle X_n, Y_n - Y \rangle | + |\langle X_n - X, Y \rangle | \\ 
\leq ||X_n|| ||Y_n - Y|| + ||X_n - X|| ||Y||.
\end{align*} 

We notice 
\[ ||X_n - X||||Y||\rightarrow 0, \]  because the norm is finite. Similar for the other term (we know the $ \sup_n X_n$ is bounded, so we can use the same argument). 
\endproof


\begin{exercise}
Use 
\ref{eq3:1}
 to show $X_n \overset{L_2}{\rightarrow} X \Rightarrow X_n \overset{L_1}{\rightarrow} X$
\end{exercise}


\begin{exercise}
Show $X, Y \in L_2$ then
\[ \Sd(X + Y) \leq \Sd(X) + \Sd(Y) . \]
\end{exercise}

\section{Conditional expectation}

$ \Fc_0 \leq \Fc$ sub $\sigma-$algebra then
$L_2(\Omega, \Fc_0, P) \leq L_2(\Omega, \Fc, P)$ as a closed subspace.


\begin{definition}
The projection of $X \in L_2(\Omega, \Fc, P)$ on  $L_2(\Omega, \Fc_0, P)$ is called \New{ conditional expectation} of $X$ with respect to $\Fc_0$, notation $\E[X | \Fc_0]$.
\end{definition}

\begin{definition} (alternative)
The conditional expectation of nonnegative or integrable $X$ with respect to $\Fc_0$ is a $\Fc_0-$measurable random variable denoted by $\E[X | \Fc_0]$ such that 
\[ \int_A \E[X | \Fc_0] dP = \int_A X dP, \;\; \forall A \in \Fc_0. \]
\end{definition}

\begin{observation}
The difference is the second definition is more general, as it asks for less conditions. There is also a subtle difference:
According to definition one
$\E[X | \Fc_0] \in  L_2(\Omega, \Fc, P)$ and according to definition two 
$\E(X | \Fc_0)$ is one of many equivalent random variables. 

This integral exists because of the Radon-Nykodimov theorem. For $X \in L_2$ two definitions are (essentially) equivalent.
\end{observation}

\begin{example}
Assume $X \in  L_2(\Omega, \Fc, P)$ and $Y = E(X |\Fc_0)$ with respect to definition one. Then $Y$ also satisfies the second definition.

Really all we need to show  is $Y$ satisfies the integral, which is equivalent to the 
\[ \int_A(X-Y)\mathbbm{1}_A dP = 0, \]
or equivalently
\[ \E[X - Y] \mathbbm{1}_A =  \langle X-Y, \mathbbm{1}_A \rangle = 0  .\]
\end{example}

\begin{definition}
For nonnegative or integrable $X$ we know
\[ \E[X | Y ] = \E[X | \sigma(Y)] . \]
$Y$ is an arbitrary random vector on the same probability space. 
\end{definition}


\begin{example}
\begin{enumerate}
\item $\Fc_0 = \{\emptyset, \Omega\}$. Clearly $\E[X | \Fc_0] = \E[X]$,
\item If $X \in \Fc_0$ measurable, then $\E[X | \Fc_0] = X$.
\end{enumerate}
\end{example}

\begin{theorem} \Important{(properties of conditional expectation)}
For $X,Y$ integrable random variables on $(\Omega, \Fc, P)$ and $\Gc \subset \Fc  , \;\sigma-$algebra. The following holds:
\begin{enumerate}
\item \[ \E[\alpha X + \beta Y | \Gc] = \alpha \E[X | \Gc] + \beta \E[Y | \Gc] . \]
\item if $Z$ is $ \Gc $ measurable, $Z, X \in L_2$ then 
\[ \E[Z X|\Gc] \overset{a.s.}{ = } Z \E[X|\Gc]  . \]
\item \[ X \geq 0  \Rightarrow  \E[X | \Gc] \overset{a.s}{ \geq } 0 \]
\item  $ \Gc_0 \subset \Gc \subset \Fc \sigma-$algebra
\[ \E \left[ \E[X|\Gc] |\Gc_0\right] \overset{a.s.}{=} \E[X|\Gc_0] . \]
In particular for $\Gc_0 = \{\emptyset, \Omega\}$ we see
\[ \E\left[\E[X|\Gc]\right] = \E[X] . \]
\item if $X$ is independent of $G$ then
\[ \E[X | \Gc ] = \E[X] . \]
\end{enumerate}
\end{theorem}

\begin{lemma} \Important{(Dudley)}
For random variables $(Y_\alpha: \alpha \in A)$. If $X \in \sigma(Y_\alpha: \alpha \in A)$ then
\begin{enumerate}
\item $|A| = k < \infty \Rightarrow$ there exists a measurable $g: \R^k \rightarrow \R$ and 
\[X = g(Y_{\alpha_1}, \dots, Y_{\alpha_k}) . \]
\item $|A| = +\infty \Rightarrow$ there exists a countable set of indices $\{\alpha_i\}_{i \in \N} \in A$ and measurable
$g: R^\N \rightarrow \R$ such that 
\[ X = G(Y_{\alpha_1}, \dots) . \]
\end{enumerate}
\end{lemma}

Since $\E[X|Y] \in \sigma(Y) \Rightarrow \E[X|Y] = g(Y)$ for some $g$. It makes sense to write $\E[X | Y = y]$. 

\section{Linear Predictors}

Assume: $(X_t)$ is weakly stationary with mean $0$. 

\begin{definition}
\New{ The best  linear predictor} for $X_{n+1}$ in terms of $X_1, \dots, X_n$ is the linear combination 
\[ Z = \varphi_1 X_n + \dots + \varphi_1 X_1 \]
such that 
\[ \E\left[ (Z - X_{n+1})^2\right] = \inf_{y \in 
\Span(X_1, \dots, X_n)} \E\left[(X_{n+1} -Y)^2 \right]  . \]
\end{definition}

Notice that $ \Span(\dots)$ is a closed subset of $L_2(\Omega, F, P)$. The projection theorem implies such a $Z$ exists and is unique! 

We denote $M_n = span(X_1, \dots, X_n)$ and $\Pi_n = \Pi_{M_n}$ i.e. $\Pi_nX_{n+1} = \varphi_1 X_n + \dots + \varphi_n X_1 = Z$. 

\begin{observation}
One should write $\varphi_{n 1}, \dots, \varphi_{n n}$.
\end{observation}

The projection theorem implies 
\begin{align*} Z \in M_n, X_{n+1} - Z \perp M_n \iff \\
 Z \in M_n, <X_{i+1} - Z, X_i> = 0, i = 1, \dots, n \end{align*}
 and that is equal to
\begin{align*}
Z \in M_n, \langle X_{n+1}, X_i \rangle = \langle Z, X_i \rangle,  i = 1, \dots, n \\
 = \langle \varphi_1 X_n+ \dots, X_i \rangle = \varphi_1 \langle X_n, X_i \rangle + \dots 
\end{align*}

but $\langle X_j, X_i\rangle = \E[ X_j X_i] = \Cov(X_j, X_i) = \gamma_X(j-i)$, which is equivalent to
\[ Z \in M_n, \gamma_X(n + 1 -i) = \varphi_1 \gamma(n - i) + \dots, i = 1, \dots, n , \]
 which is just a linear system of equations:

\[  Gamma_n  
\left( \begin{array}{c} \varphi _1 \\ \varphi _2 \\ \vdots \\ \varphi _n \end{array} \right)
 = 
\left( \begin{array} {c c c c} 
 \gamma(0) & \gamma(1)  & \ldots & \gamma(n-1) \\
\gamma(1) & \gamma(0) & \gamma(1) \ldots & \gamma(n-2) \\
\vdots & \ddots & \ddots & \vdots \\
\gamma(n-1) & \ldots & \gamma(1) & \gamma(0)
\end{array} \right)
\]



This system has a solution by projection theorem. But not necessarily unique unless $\Gamma_n$ is regular, which is equivalent to begin positive definite. 

The \New{ prediction error} equals 
\[  \E \left[ \left( X_{n+1} - Z \right) ^2 \right]    = \E \left[ (X_{n+1 } - \Pi_n X_{n+1})^2   \right] .  \]

Observe:
\begin{align*} ||X_{n+1}||^2  = & \\ 
= & \langle X_{n+1} - \Pi_n X_{n+1} + \Pi_n X_{n+1}, X_{n+1} - \Pi_n X_{n+1} + \Pi_n X_{n+1} \rangle \overset{orthogonality}{=}  \\
= & ||X_{n+1} - \Pi_n X_{n+1}||^2 + ||\Pi_n X_{n+1}||^2 
\end{align*}

 (basically Pythagorean theorem for $L_2 $.

The error is therefore:
\begin{align*}
 \E\left[(X_{n+1} - \Pi_n X_{n+1})^2\right]  \\ 
= \E\left[ (X_{n+1})^2\right] - E\left[\Pi_n X_{n+1}\right] ^2  \\ 
= \gamma_X (0) - \langle \varphi_1 X_n + \ldots \varphi _n X_1, \varphi _1 X_n + \ldots \varphi _n X_1 \rangle \\
= \gamma_X(0) - \sum_{i=1} ^n \sum_{j=1} ^n  \varphi_i \varphi_j \gamma(i-j)    \\
 = \gamma_X(0) - \left<  \left( \begin{array}{c} \varphi _1 \\ \vdots \\ \varphi _n \end{array}  \right) ,  \Gamma_n   \left( \begin{array}{c} \varphi _1 \\ \vdots \\ \varphi _n \end{array} \right) \right>  \\
 = \gamma_X(0) - \left<  \left( \begin{array}{c} \varphi _1 \\ \vdots \\ \varphi _n \end{array}  \right) ,     \left( \begin{array}{c} \gamma (1) \\ \vdots \\ \gamma(n) \end{array} \right) \right>  
\end{align*}

\begin{exercise}
Show that the prediction error for $X_{n+h}, h \geq 2$ equals 
\[ \E \left[ (X_{n+h}- \Pi_nX_{n+1})^2 \right]  = \gamma(0) - \left< \left( \begin{array} {c} \alpha _1 \\ \vdots \\ \alpha _n \end{array} \right) , \left( \begin{array} {c} \gamma(h) \\  \vdots \\ \gamma(n+h) \end{array} \right) \right> ,\]
where 
\[ \Pi_n X_{n+h} = \alpha_1 X_n + \dots + \alpha_n X_1, \]
which satisfy 
\[ \Gamma_n \left( \begin{array} {c} \alpha_1 \\ \vdots \\ \alpha _n \end{array} \right) 
 = \left( \begin{array} {c} \gamma(h) \\ \vdots \\ \gamma(n+h-1) \end{array} \right)  \]
\end{exercise}

\begin{example} ($\Ar(1)$ process)
For $(Z_t) \sim WN(0, \sigma^2)$ suppose a stationary sequence $(X_t)$ satisfies
\[ X_t = \varphi X_{t-1}, t\in \Z, |\varphi| < 1, \]  then
\[ X_t = \sum_{i = 0}^\infty \varphi^i Z_t , \]  and
$Z_{n+1} \perp X_j, j=1,\dots,n$ (because $Z_{n+1} \perp Z_j, j\leq n$ and the scalar product is continuous).

\[ \Pi_n X_{n+1} = \Pi_n(\varphi X_n) + \Pi_n(Z_{n+1}) = \varphi X_n + 0 \]
 ($Z_{n+1}$ is orthogonal to previous guys, so projection is $0$). This is not true if $|\varphi| > 1$.

\end{example}

\begin{exercise}
Show in the ($\Ar(1)$) example that 
\[ \Pi_n X_{n+h} = \varphi^h x_n, h \geq 1. \]
 The prediction error in $\Ar(1)$ example:
\[ \E \left[ (X_{n+1} - \varphi X_n)^2 \right] = \E \left[ (Z_{n+1})^2 \right] = \sigma ^2 . \]
\end{exercise}

\begin{example} (periodic process)
\[ X_t = A_1 \cos{\nu t} + A_2 \sin{\nu t}, \] 
 $  \Var( A_1) = \Var (A_2) , \E[A_1] =  \E[A_2] = 0, \Cov(A_1, A_2) = 0$.
Then 
\[ \Pi_n X_{n+1} = 2 \cos{v} X_n - X_{n-1} . \]
 Moreover 
$ \gamma(h)  = \sigma^2 \cos{vh}$.  

\begin{align*}
 \E\left[(X_{n+h} - Pi_nX_{n+1})^2\right] = \sigma^2 - \left<\left( \begin{array}{c} 2 \cos{\nu} \\ -1 \\ 0 \\ \vdots \\ 0 \end{array} \right), \left( \begin{array} {c} \sigma^2\cos{\nu} \\ \sigma^2 \cos{2\nu} \vdots \\ \sigma^2 \cos{n v} \end{array} \right) \right> = \\
= \sigma^2 - (2 \sigma^2 \cos^2{v} - \sigma^2 \cos{2v}) = 0 . \end{align*}

Here $\varphi$ -s are not unique! 
\end{example}

The problem is that in practice we do not know $\gamma$ -s, which we used in the examples.

\section{Case of non-zero expectations}

Assume $(X_t)$ weakly stationary, $ \E[ X_t] = \mu$.
\[ M_n' = \Span(1, X_1, \dots, X_n) \]

\Important{The best linear predictor} of $X_{n+1}$ in terms of $1, X_1, \dots, X_n$ is the projection of $X_{n+1}$ on $M_n'$.

If $\mu = 0 = \E[ X_t] $ then 
\[ \Pi_{M_n'}X_{n+h} = \Pi_n X_{n+h} \]

Assume $Y_1, \dots, Y_n \in L_2$ and $\E[Y_1] = \dots = \E[Y_n] = 0$. It follows that $Y \perp 1, i = 1, \dots, n$.  Then the 
\[ \Span(1, Y_1, \dots, Y_n) = \Span(1) +_{\perp}  \Span(Y_1, \dots, Y_n). \]

If $ \E[ X_t] = \mu$, we can work with  $ X_t - \mu$. 
\[ \Pi_{M_n'}(\mu + X_{n+1} - \mu) = \mu + \Pi_{\Span(X_1 - \mu, \dots , X_n - \mu)}(X_{n+1}- \mu) . \]
($M_n' = \Span(1) +_{\perp} \Span(X_1 - \mu, \dots)$)

Now we have a weakly stationary series of mean $0$, so we know how to do it, if we estimate $\mu$. 

\begin{observation}
If we have a stationary sequence with mean $\neq 0$ we can:
\begin{enumerate}
\item subtract the mean to get $Y_t = X_t - \mu$
\item predict $Y_{n+h}$ in terms of $Y_i$
\item add $\mu$ back to get the prediction for $X_{n+h}$
\end{enumerate}
\end{observation}

\section{ Non-linear prediction}

\begin{definition}
\New{ The best predictor} for $X_{n+1}$ in terms of $X_1, \dots, X_n$ is a random variable $ f_n(X_1, \ldots , X_n)$, which minimizes 
\[ \E[(X_{n+1}- f_n(X_i))^2] , \]
 in the class of all measurable functions $f_n: \R^n \rightarrow \R$. 
\end{definition}

In $L_2(\Omega, \Fc, P)$ space by definition one of conditional expectation the best predictor is
\[ \E[X_{n+1} | X_1, \dots, X_n] , \]
 or 
\[ \E[W | X_1, \dots, X_n] , \]
 if we want to predict $W$. 

\begin{example} (\New{ARCH(1)})

Suppose $0< \alpha_1 < 1$. The ARCH(1) equation has a causal stationary solution in the sense:
there exists a stationary process 
\[ X_t = \sigma_t Z_t, Z_t \sim IID(0,1) , \]
 and $\sigma_t^2 = \alpha_0 + \alpha_1 X_{t-1}^2$. 

Then 
\[ \E[X_{n+1} | X_1, \dots, X_n] = 0. \] 
 Even $ \E[X_{n+1} | X_j, j =-\infty, \dots, n] = 0$.

\begin{align*}
\E[X_{n+1} | X_1, \dots, X_n ] = \E \left[ \E[X_{n+1}|\sigma_j, X_j, j = -\infty, \dots, n] |X_1, \dots, X_n \right] = \\
=  \E\left[\E[\sigma_{n+1} Z_{n+1}|\sigma_j, X_j, j = -\infty, \dots, n] |X_1, \dots, X_n \right] = \\
=  \E\left[ \sigma_{n+1} \E[Z_{n+1}|\sigma_j, X_j, j = -\infty, \dots, n]|X_1, \dots, X_n \right] . 
\end{align*}
 (because $Z_{j+1}$ is independent of the others the best we can predict is the average, which is $0$).

\[ = \E[\sigma_{n+1} 0] = 0. \]
\end{example}

\begin{example} ($\Ar(1)$)
Suppose $|\varphi| < 1, Z_t \sim IID(0,1)$ 
\[  X_t = \varphi X_{t-1} + Z_t, t\in \Z , \]
 and $X_t = \sum_{j = 0}^\infty \varphi^j Z_{t-j}$.

\[ \E[X_{n+1}|X_1, \ldots , X_n] = \E[ \varphi X_n + E(Z_{n+1} | X_1, \ldots, X_n] = \varphi X_n. \]

It follows that the best predictor is the best linear predictor.
\end{example}

\section{ Partial autocorrelation function}

Assume $(X_t)$ is weakly stationary, mean $0$.
\[ \Pi_{(2, \dots, n)} = \Pi_{\Span(X_2, \dots, X_n)} . \]

Define 
\begin{align*} \alpha(1) = & \Corr(X_2, X_1) = \rho(1), \\
 \alpha(h) = & \Corr(X_{h+1} - \Pi_{2, \dots, h} X_{n+h}, X_1 - \Pi_{(2, \dots, h)}X_{n+1}) \end{align*}

(we subtract other variables to see if the variables are really causal and not just correlated).

Function $\alpha_X = \alpha: \N \rightarrow [-1,1]$ given above is called the \New{ partial autocorrelation function} of the sequence $(X_t)$. We call it \New{ pacf}.

If $\E[X_t ] = \mu$ we define 
\[ \Pacf(X_t) = \Pacf(X_t - \mu). \]

\begin{example} ($\Ar(1)$)
Assume that $(X_t)$ is a stationary process with mean $0$ and satisfies 
\[ X_t = \sum_j \varphi^j Z_{t-j} , \]
 and 
\[ X_t = \varphi X_{t-1} + Z_t. \]

Then 
\[ \alpha(1) = \Corr(X_2, X_1) = \Corr(\varphi X_1 + Z_2, X_1) = \varphi . \]

$h \geq 2$:
\[ \Pi_{(2,\dots,n)}X_{n+1} = \varphi X_n. \]
But here 
\begin{align*}  \alpha_X(h) = \Corr(X_{n+1} - \varphi X_n, X_1 - \Pi_{(2, \dots,h)}X_1)  \\ 
 =  \Corr(Z_{n+1}, X_1 - \Pi_{(2, \dots,h)}X_1) = 0 \end{align*}
  (because $Z_{n+1}  \perp \Span(X_1, \dots, X_n)$, which is the right part. Also the scalar product is continuous).


\begin{lemma}
Assume $(X_t)$ is weakly stationary with mean $0$., satisfying the following (mild) conditions: $\gamma(0) > 0$ and $\gamma(h) \rightarrow 0$ for $ |h| \rightarrow   \infty$. 

\[ \Pi_h X_{h+1} = \varphi_{h,1} X_h + \dots + \varphi_{h,h} + X_1\]
 is the best linear predictor for $X_{h+1}$ in terms of $X_1, \dots, X_h$. Then
\[ \alpha_X(h) = \varphi_{h,h}. \] 
\end{lemma}

Let us continue the example. 

The best linear predictor is in this example 
\[ \Pi_h X_{h+1} = \varphi X_h , \] $(h\geq 2)$.

The lemma 
implies 
\[ \alpha_X(h) =  \begin{cases} \varphi, & h =1 , \\  0 , & \text{otherwise} \end{cases} , \] $(h \geq 2)$.
\end{example}




This is important because of duality:

\begin{example} (MA(1))
Assume $Z_t \sim WN(0, \sigma^2),$ 
\[  X_t = Z_t + \theta Z_{t-1}. \]
Then 
\[ \rho_X(h) = \begin{cases} \frac{v}{1+v^2} , &   h = 1 , \\ 0 , & \text{otherwise} . \end{cases} \]

We can see this in the autocorrelation plot. In practice we estimate it and if they are close to $0$ and can conclude we have MA(1). 
\end{example}

\begin{lemma}
Assume $(X_t)$ weakly stationary, mean $0$ such that $\gamma(0) > 0$ and $\gamma(h) \rightarrow 0$, $h \rightarrow \infty$ then $\Gamma_n$ is regular for every $n \in \N$. 
\end{lemma}

Regularity ensures there is only one solution in the previous lemma. Meaning $\varphi_{n,1}, \dots, \varphi_{n,n}$ are unique. 


\section{ Spectral representation of time series}

Assume $(X_t)$ is a $\C$ valued time series.
\[ X_t = X_t^{(1)} + i X_t^{(2)} . \]

\begin{align*} \E[X_t] = & \E[X_t^{(1)}] + i \E[X_t^{(2)} ] , \\ 
\gamma(h) = & \Cov(X_{t+h}, X_t) = \E[ (X_{t+h} - \mu)\overline{(X_t - \mu)} ] \end{align*}

Assume $(X_t)$ are stationary. $\gamma: \Z \rightarrow \C$. $\gamma$ is hermitian, 
\[ \gamma(h) = \overline{\gamma(-h)}. \]  $|\gamma(h)| \leq \gamma(0) \in [0, \infty)$. $\gamma(h)$ is positive semidefinite i.e.
\[ \sum_{i,j = 1}^n \alpha_i \gamma(i-j)\overline{\alpha_j} \geq 0. \]

\begin{observation}
$\gamma$ is an autocovariance function iff these properties hold (the first two).
$X_t^{(2)}$ identically equals $0$, nmakes $\R$ values time series special case. 
\end{observation}

\begin{theorem} (\Important{Hergoltz})
Function $(\gamma(h))_{h \in\Z}$ is positive semidefinite hermifian $\iff $ there exists a finite measure $\nu$ on $(-\pi, \pi]$ such that 
\[ \gamma(h) = \int_{(-\pi,\pi]} e^{i h u}\nu(du). \] 
Such a $\nu$ is unique.
\end{theorem}

Observe $\nu$ has its own distribution function $F : (-\pi, \pi] \mapsto \R_+$, 
\[ F(x) := \nu((-\pi,x]) . \]

% slika distribucije. Ima končno mnogo skokov, ker je mera končna. ne rabi končat pri 1. 

If $\nu$ is absolutely continous with respect to Lebesgue measure, than there exists a \New{ spectral density} $f_X$, such that \[ \nu(-\pi, x] = \int_{(-\pi,x]}f_X(x) dt. \]
 Then 
\[ \gamma(h) = \int_{-\pi}^\pi e^{i h u}f_X(u) du , \; h = 1,2, \dots. \]

\begin{theorem} (about spectral density)
Assume $(X_t)$ is weakly stationary time series, such that $\sum_{h =-\infty }^\infty|\gamma(h)| < \infty$. Then $\nu$ has spectral density. Moreover 
\[ f_X(\nu) = \frac{1}{2 \pi} \sum_{h = -\infty}^\infty\gamma(h) e^{-i h u}, \]
$ u\in (-\pi, \pi]$ (inverse Fourier transform). Note that $f_X(\nu) \in [0, \infty)$. 
\end{theorem}
We call such time series \Important{short range dependant} time series. 

Variance of the series if nothing but the integral of the spectral density: 
\[ \gamma(0) = \int_{-\pi}^\pi f_X(u)du. \]

\begin{example}
\begin{enumerate}
\item $Z_t \sim WN(0, \sigma^2)$, 
\[ \gamma_Z(h) = \begin{cases} \sigma^2, & h = 0 \\ 0, & \text{otherwise} \end{cases} . \]
 One can directly show the spectral density exists. 
\[ f_Z(h) = \frac{\sigma^2}{2 \pi}, \]
 which is a uniform density. 

White noise contains all frequencies $u \in (-\pi, \pi]$ with same intensity! 
\item  
\[ X_t = A e^{i t u}, \] 
$ \E[ A^2 ] = \sigma^2$. Direct calculation shows 
\[ \nu = \sigma^2 \delta_u \] (Dirac measure). The period = $2 \pi / u$, where $u$ is the only frequency. 
\end{enumerate}
\vskip0.5cm

If $(X_t)$ and $(Y_t)$ are uncorrelated then  $(X_t + Y_t)$ has the spectral measure 
\[ \nu = \nu_X + \nu_Y. \]
 The same holds for their spectral density functions if they exist. 

Assume $(X_t)$ is weakly stationary, $\sum_{j=-\infty}^\infty |h_j| < \infty \Rightarrow $ 
\[ Y_t = \sum_j h_j X_{t-j} \]
 is well defined and weakly stationary. We can define the response function 
\[ A(e^{-iu}) = \sum_j h_j e^{-u i j}. \]

If $(X_t)$ has the spectral density $f_X$ then $(Y_t)$ has the spectral density 
\[ f_Y(u) = |A(e^{-iu})|^2 f_X(u) . \] 

\begin{theorem} (\Important{Cramer})
Assume $(X_t)$is weakly stationary with mean $0$. Then there exists a random process $(Z(u))_{u \in(-\pi, \pi]}$ with 
orthogonal increments (similar to independent in Brownian motion) and values in $\C$. 
\[ X(t) = \int_{-\pi}^\pi e^{i t n}dZ(u) . \]
Moreover 
\[ \Var(Z(u_2) - Z(u_1)) = \nu(u_1, u_2], \; u_1, u_2 \in (-\pi,\pi]. \]

This is a stochastic integral:
$\int_a^bdZ(u) = Z(b) - Z(a)$.
\end{theorem}

Continue example.
\[ Z(w) = \begin{cases} 0,  & w < n , \\ A , \text{otherwise} \end{cases} . \] 

\[ \Var(Z(u_2) - Z(u_1)) = \sigma^2 \delta_u(u_1, u_2]. \]

The natural estimator for $f_X$ is defined by 
\[ w_j = \frac{2 \pi j}{n} \in (-\pi, \pi]. \]
\[ \hat{f}_X(w_j) =\frac{1}{2\pi} I(w_j)= \frac{1}{2\pi} \sum_{|k| < n} \sqrt{\hat{\gamma}(k)e^{-ikw_j}} . \]

This is not a consistent estimator. To make it more consistent we smooth it. 

This is a {\bf periodogram}. We try to look at the time of time series. But we could look at frequencies (engineers do it a lot). And it is the same (dual). 
\end{example}









$ \Pacf \to \Ar $

ocf  $\to  \Ma $

$\Acf $

arima.sim

par $->$ splits plot into more parts

qqmorm $->$ Normal Q-Q plot

spec $->$ for spectral analysis

spec.pgram $->$ smoothed periodogram

periodogram

\begin{observation}
For $\R$-valued time series $f_X$ is symmetric (because of Fourier)! The frequencies are also divided by $2 \pi$ and that is why the interval we draw is $[0, 1/2]$.
\end{observation}

library TSA

Check distribution smoothing. 

locator $->$ reading points from a graph

which.max

You can also use this (frequencies) to detect the seasonality of the data.

To test which peaks are significant is quite hard. 


\chapter{Large sample theory (estimation of $\mu, \gamma, \rho$ and $ \alpha$)}



\[ {\bf X} _n = (X_{n, 1}, \dots, X_{n,k}) : \Omega \rightarrow \R^k \]
 random vectors, $n \in \N . $

Recall ${\bf X}_n \overset{D}{\rightarrow}  {\bf X}$ if 
\[ P(X_n \leq {\bf x}) \rightarrow P({\bf X} \leq {\bf x}) , \]
 for all ${\bf x} \in \R^k$ in which ${\bf t} \rightarrow P({\bf X} \leq {\bf t})$ is continuous. 

\[ {\bf X}_n \overset{P}{\rightarrow}  {\bf X} , \]
 if for all $\epsilon > 0$ 
\[  \lim_{n \rightarrow \infty} P(|X_n - {\bf X}| > \epsilon) = 0. \]
 
\[ {\bf X}_n \overset{L_p}{\rightarrow} {\bf X} \]
if 
\[ ||X_n - {\bf x}||^p \rightarrow 0 . \]

\[ {\bf X}_n  \overset{a.s.}{\rightarrow} {\bf X} , \]
 if 
\[ {\bf P}({\bf X}_n \rightarrow {\bf X}) = 1 . \]

\begin{displaymath}
 \xymatrix{
  \overset{a.s}{\to} \ar[dr]   & \overset{ L_p}{\to} \ar[d] & \overset{D}{\to}  \\
 & \overset{P}{\to} \ar[ur]  &  }
\end{displaymath}


\begin{theorem} (\Important{Portmanteau})
The following are equivalent
\begin{enumerate}
\item \[ {\bf X}_n \overset{D}{\rightarrow} {\bf X} .\] 
\item \[  \E[ f({\bf X}_n)] \rightarrow \E[f({\bf X})]  , \] for all $f: \R^k \rightarrow \R$ continuous and bounded 
\item \[ P({\bf X}_n \in B) \rightarrow p({\bf X} \in B) , \] for all $B$ Borel set in $\R^k$, $P({\bf X} \in \delta B) = 0$
\item \[ \limsup_n{P({\bf X} \in F)} \leq P({\bf X} \in F) , \] for all closed sets $F \subset \R^k$
\item \[ \liminf_n{P({\bf X} \in G)} \ni P({\bf X} \in G) , \] for every open set $G \subset \R^k$
\item \[ \E \left[ e^{i \langle t, {\bf X}_n \rangle} \right]  \rightarrow \E \left[e^{i\langle t, {\bf X} \rangle} \right] , \] for all ${\bf t} \in \R^k$
\end{enumerate}
\end{theorem}

\begin{exercise}
${\bf X}_n \overset{P}{\rightarrow} {\bf X}, {\bf Y}_n \overset{P}{\rightarrow} {\bf Y}$ on the same probability space, then 
\[ ({\bf X}_n, {\bf Y}_n) \overset{P}{ \rightarrow} ({\bf X}, {\bf Y}) . \]
\end{exercise}

\begin{exercise}
${\bf X}_n \overset{D}{ \rightarrow} {\bf X}, {\bf Y}_n \overset{D}{\rightarrow} {\bf Y}$ on the same probability space, then show by counterexample that we do {\bf not} need to have 
\[ ({\bf X}_n, {\bf Y}_n) \overset{D}{ \rightarrow} ({\bf X}, {\bf Y}) . \]
\end{exercise}

\begin{theorem} (\Important{ continuous mapping theorem})
Assume $ g: \R^k \rightarrow \R^m$ is measurable and continuous on the set $C \subset \R^k$ and $P({\bf X} \in C) = 1$, then 
\[ {\bf X}_n \overset{*}{\rightarrow} {\bf X} \Rightarrow g({\bf X}_n) \overset{*}{ \rightarrow} g({\bf X}) , \]
 for $ * = D, P, a.s.$
\end{theorem}



\begin{lemma}
Assume ${\bf X}_n \overset{D}{\rightarrow} {\bf X}, {\bf Y}_n \overset{P}{\rightarrow} C$ on the same probability space, then 
\[ ({\bf X}_n, {\bf Y}_n) \overset{D}{\rightarrow} ({\bf X}, C) .\]
\end{lemma}

\begin{exercise}
Assume ${\bf X}_n \overset{D}{\rightarrow} {\bf X}$ and ${\bf Y}_n \overset{D}{\rightarrow} {\bf C}$ on the same probability space. Then
\begin{enumerate}
\item ${\bf X}_n + {\bf Y}_n \overset{D}{  \rightarrow} {\bf X} + {\bf C}$ for $k = m$
\item ${\bf X}_n  {\bf Y}_n \overset{D}{ \rightarrow } {\bf X}  {\bf C}$ for $k = m$ or $k = 1$ or $m = 1$
\item ${\bf X}_n / {\bf Y}_n \overset{D}{ \rightarrow} {\bf X} / {\bf C}$ for $k = 1$ and $c \neq 0$
\end{enumerate}
\end{exercise}

\begin{proposition} (\Important{1st about approximation})
${\bf X}_n \overset{D}{\rightarrow} {\bf X}, {\bf X}_n - {\bf Y}_n \overset{P}{\rightarrow} 0 \Rightarrow {\bf Y}_n \overset{P}{ \rightarrow } {\bf X}$.
\end{proposition}
\proof
\[ {\bf Y}_n = {\bf X}_n + ({\bf Y}_n - {\bf X}_n) \overset{D}{ \rightarrow} {\bf X} + 0 , \]
 and this is enough according to the previous exercise.
% \ref
\endproof

\begin{exercise}
\[ {\bf Y}_n \overset{D}{ \rightarrow} C \iff {\bf Y}_n \overset{P}{ \rightarrow} C\]
 for $C \in \R^n$.
\end{exercise}

\begin{proposition} (\Important{2nd about approximation})
Assume 
\begin{enumerate}
\item \[ {\bf Y}_{n,j} \overset{D}{\rightarrow} {\bf Y}_j, n \rightarrow \infty \]
\item \[ {\bf Y}_j  \overset{D}{\rightarrow} {\bf Y}, j \rightarrow \infty \]
\item \[ \lim_j \limsup_{n\rightarrow \infty}{P(|X_n - Y_{n,j}| > \epsilon)} = 0 \] for all $\epsilon > 0.$
\end{enumerate}
Then 
\[ {\bf X}_n \overset{D}{\rightarrow} {\bf Y}. \]
\end{proposition}
\proof
For all $F $ closed 
\[ \limsup_n{P({\bf X}_n \in F)} \leq P({\bf Y} \in F) , \]
 is enough. 
\[ F^\epsilon = \{y | \inf_{x\in F}||{\bf y} - {\bf x}|| \leq \epsilon \} , \]
 closed set. Then 
\[ P({\bf X}_n \in F) \leq P(Y_{n j} \in F^\epsilon) + P(||Y_{n j} - X_n|| > \epsilon). \]
 We take 
\[ \limsup{P({\bf X}_n \in F)} \leq P(Y_j \in F^\epsilon) + \limsup{P(|X_n - Y_{n j}| > \epsilon)}. \]
 We apply $\limsup_{j\to \infty}$ and get 
\[ \limsup{P({\bf X})_n \in F} \leq P(Y \in F^\epsilon). \]

If we let $\epsilon \rightarrow 0$ then the right side converges to 
\[ P(Y \in F) , \] because the probability is continuous with respect to decreasing the segment of sets 
\[ \bigcap_{\epsilon > 0} F^\epsilon = F. \]
\endproof

\begin{example}
Assume estimator $T_n, S_n$ satisfy 
\[ \sqrt{n}(T_n - \theta) \overset{D}{\rightarrow} N(0, \sigma^2) , \]
 and 
\[ \sqrt{n}\frac{T_n - \theta}{S_n} \overset{D}{\rightarrow} N(0,1) \] (by exercise \ref{ex:63}).
Then approximately $(q - \alpha) 100 \%$ confidence interval for $\theta$ is $(T_n - \frac{S_n}{\sqrt{n}}Z_{\alpha/2},T_n + \frac{S_n}{\sqrt{n}}Z_{\alpha/2})$.
\end{example}

\begin{theorem} (\Important{Strong law of large numbers} )
Assume $(X_n)$ iid with finite mean $\mu$. Then 
\[ \overline{X}_n = \frac{1}{n} \sum _{i=1}^n X_i \overset{a.s.}{ \rightarrow}\mu . \]
\end{theorem}


\begin{theorem} (\Important{Central limit theorem - Levy})
Assume $(X_n)$ iid with finite second moment $\sigma^2$. Then 
\[ \sqrt{n}(\overline{X} - \mu)  \overset{D}{\rightarrow} N(0, \sigma^2). \]
\end{theorem}


\begin{observation}
This also holds for random vectors. There is a stronger (\Important{Lindeberg}) version of the theorem, that allows
 $(X_n)$ to be independent but not necessarily equally distributed.
\end{observation}


The question is: What if we have dependance? Can we still have the CLT?

Time series $(X_t)$ is \New{ $m-$dependent}if for all $t \in \Z$ families of random variables $\{\dots, X_{t-1}, X_t\}$ and $\{X_{t + m + 1}, \dots\}$ are independent. 

\begin{exercise}
If $(Z_t) \sim IID$ and $X_t = Z_t + \theta Z_{t-1}$ is a MA(1) process, show $(X_t)$ is $1-$dependent.
\end{exercise}

\begin{exercise}
Show $(X_t)$ is $0-$dependent $\iff$ $X_t$-s are independent random variables.
\end{exercise}

\begin{theorem} (\Important{CLT for m-dependent sequences})
Assume $(X_t)$ is strictly stationary (otherwise we have no chance), m-independent with mean $0$, and with finite variance. Then 
\[ \sqrt{n}\;\overline{X}_n \overset{D}{ \rightarrow} N\left(0, \sum_{n = -m}^m \gamma_X(h) \right) . \]
\end{theorem}
Ask if $\overline{X}_n \overset{P}{ \rightarrow} 0$ and how fast. Then 
\begin{align*} \Var(\sqrt{n} \; \overline{X}_n) = \Var\left(\frac{X_1 + \dots + X_n}{\sqrt{n}}\right)  \\
= \frac{1}{n} \sum_{i,j = 1}^n \Cov(X_i, X_j) = \frac{1}{n} \sum_{i,j = 1}^n \gamma(i-j) \\ 
= \frac{1}{n} \sum_{h  = -n + 1}^{n-1}(n - |h|)\gamma_X(h). \end{align*}

Assume $\sum_{h=-\infty}^\infty |\gamma(h)| < \infty$. Then the above equation $\rightarrow \sum_h \gamma_X(h)$ as $n \rightarrow \infty$. We see this by the dominated convergence theorem (we can therefore swap $\lim$ and $\sum$). 

By the Chebyshev inequality $X_n$ is really converging to $0$ in probability: 
\[
 P(|\overline{X}_n - 0| > \epsilon ) \leq \frac{1}{n} \frac{Var \overline{X}_n}{\epsilon^2}   \rightarrow 0 
\]

\proof
Note $\gamma(h) = 0$ for $|h| > n$ (there is no dependance). We showed 
\[ \sqrt{n} \; \overline{X}_n = n \Var \overline{X}_n \rightarrow \sum_{h=-m}^m \gamma(h). \]
 Fix $k > 2m$, $r = \lfloor n/k \rfloor$. $r k \leq n, n - rk < k$. 

$\overline{X}_n = $ average of $X_i, i = 1,\dots,n$.

\begin{align*} X_1, \dots, X_{k-m}, X_{k-m+1}, \dots, X_{k}, X_{k+1}, \dots, X_{2k-m}, \\
 X_{2k-m+1},\dots,X_{2k}, \dots \dots, X_{rk-m+1}, \dots, X_{r_k}, X_{rk+1},\dots,X_{n}. \end{align*}

 The sizes of these parts with $\dots$ are $k-m, m, k-m, \dots, m, \leq k-1$ if $ k \not | n$. 
We denote them with 
\[ B_{k,1}, R_{k,1}, \dots, B_{k, r}, R_{k, r}. \]
 We introduce 
\[ Y_{n,k} = \frac{1}{\sqrt{n}} (B_{k, 1} + \dots, B_{k,r}). \]
 \[ \sqrt{n} \; \overline{X}_n = \frac{1}{\sqrt{n}} (X_1 + \dots X_n) = Y_{n, k} +  \text{remainder} , \]
 which equals the sum of $R_{k,j}-$s divided by $\sqrt{n}$. Note that $B_{k,i}-$s are IID with finite 
variance (calculated before the proof)!! 

Because of this 
\[ Y_{n,k} = \frac{1}{\sqrt{n}} \sqrt{kr} \frac{1}{\sqrt{k}} \frac{1}{\sqrt{r}} (B_{k,1} + \dots B_{k,r}). \]
 The 
\[  \frac{1}{\sqrt{r}} (B_{k,1} + \dots B_{k,r}) \rightarrow^d N(0, Var B_{k,1}) \] (all have the same variance). The 
\[ \frac{1}{\sqrt{n}} \sqrt{kr} \rightarrow 1. \] 
\[ Var B_{k,1} = Var(X_1, \dots + X_{k-m}) = \sum_{|h| \leq m}(k-m-|h|)\gamma(h) . \]

Together we get 
\[ Y_{n,k} \rightarrow N(0, \sum_{|h| \leq m}(\frac{k-m-|h|}{k})\gamma(h)) := V_k. \]
 When we let $k \rightarrow \infty$ each of 
\[ (\frac{k-m-|h|}{k} \rightarrow 1. \]
 Therefore 
$ Y_k \overset{D}{\rightarrow} N(0, V)$ as $k \rightarrow \infty$ where 
\[ V = \sum_{n = -m}^m \gamma(h). \] 

It remains to show that 
\[ \sqrt{n}\; \overline{X}_n \approx Y_{n,k} , \]
 in the sense of preposition (5). 
\[ \lim_k \limsup_n P(|\sqrt{n}\; \overline{X}_n - Y_{n, k}| > \epsilon) \leq \lim_k \limsup_n \frac{Var(1/\sqrt{n}(R_{k,1} + \dots R_{k,r}))}{\epsilon^2} . \]
 $R_{k,i}$ are independent as well and similar to $B_{k,i}$. So 
\[ \Var(R_{k,i}) = \sum_{|h| \leq m}(m- |h|)\gamma(h) , \]
 for $r = 1, \dots, r-1$. 
\[ Var(R_{k,r}) = \sum_h ((n -rk) + m - |h|)\gamma(h). \]

The limit
\begin{align*} \lim_k \limsup_n \frac{Var(1/\sqrt{n}(R_{k,1} + \dots R_{k,r}))}{\epsilon^2} \\
 \leq \lim_k \limsup_n (r-1) \sum_h \frac{m \gamma(h)}{n} + \sum_h \frac{m+k}{n}|\gamma(h)|.
\end{align*}
 Both things tend to $0$. Therefore $\sqrt{n}\; \overline{X}_n \rightarrow^d Y$. 
\endproof


All our theorems in the sequel can be proved using a similar (same) argument (called \New{ large block/small block}.

Assume $(X_t)$ satisfies the conditions of the theorem and $\E[ X_t] = \mu \neq 0$. It is clear that 
\[ Y_t = X_t - \mu \] 
satisfies the same conditions and 
\[ \sqrt{n} (\overline{X}_n - \mu) = \sqrt{n}\; \overline{Y}_n \rightarrow^d N(0,V). \]
 Observe 
\[ \gamma_X(h) = \gamma_Y(h). \]
 This is how we use the theorem. $v = \sum_{|h| \leq m} \gamma(h)$.


Now we know a CLT, that does not involve IID data. So $(1 - \alpha) 100 \%$ confidence interval for $\mu$ is 
$(\overline{X}_n +- \sqrt{\frac{v}{n}}Z_{\alpha/2})$. 

\section{Estimating the mean of MA(q) processes}

\begin{example} (MA(1))
(assume normal stuff) $\Rightarrow$ $ X_t$ is strictly stationary and 1-dependent, remember the $\gamma_X(h)$. Theorem 
(8) now proves 
\[ \sqrt{n} \; \overline{X}_n \overset{D}{ \rightarrow} N(0, \sigma^2 (1 + \theta^2 + 2\theta)) . \]
\end{example}

\begin{definition}
Assume $Z_t \sim WN$, $\theta_i$ are real numbers and 
\[ X_t = Z_t + \theta_1 T_{t-1} + \dots + \theta_q + Z_{t-q}. \]
 We ask $\theta_q \neq 0$. This is MA(q).
\end{definition}

\begin{exercise}
Show that MA(q) process has auto correlation function 
\[ \gamma_X(h) = \begin{cases} 0 , & |h| > q , \\ \sigma^2 \sum_{j = 0}^{q - |h|}\theta_k \theta_{j + |h|} , & \text{otherwise} \end{cases} \]
 where $\theta_0 = 1$. The mean of the process is $0$.
\end{exercise}

\begin{exercise}
Show that MA(q) process satisfies the following:
\[ \sum_{|h| \leq q} \gamma(h) = \sigma^2(\sum_{j = 0}^q \theta_j)^2 . \]
 In particular
\[ \sqrt{n } \; \overline{X}_n \rightarrow^d N(0, \sigma^2(\sum_{j=0}^1 \theta_j)^2), \]  
if $(Z_t) \sim IID(0, \sigma^2)$.

If 
\[ Y_t = \mu + X_t, X_t \sim MA(q) , \]
 with iid noise 
\[ \sqrt{n}(\overline{Y} - \mu) \rightarrow^d N(0, \sigma^2(\sum_{j=0}^1 \theta_j)^2). \]
\end{exercise}

\section{Estimating the mean of linear processes}

MA is just a special case of linear processes! 

\begin{proposition}
Assume $(X_t)$ is weakly stationary with mean $\mu$.
\begin{enumerate}
\item if $\gamma(h) \rightarrow 0 \Rightarrow \Var(\overline{X}_n) \rightarrow 0$ and $\overline{X}_n \rightarrow^p \mu$.
\item if $\sum_{h = -\infty}^\infty\gamma(h) < \infty \Rightarrow n \Var \overline{X}_n \rightarrow \sum_h \gamma_X(h)$
\end{enumerate}
\end{proposition}
\proof
(2) was already shown before (when showing the dominated convergence)
(1)  
\begin{align*} \Var \overline{X}_n = \frac{1}{n^2} \sum_{|h| \leq m} (n - |h|)\gamma(h) \\ 
\leq \frac{1}{n} \sum_h |\gamma(h)| \rightarrow 0, \end{align*}
 since $|\gamma(h)| \rightarrow 0$ (Cesaro mean). 

Because $\overline{X}_n$ is unbiased the Chebyshev inequality shows 
\[ P(|\overline{X}_n - \mu| > \epsilon) \rightarrow 0 . \]
$\Rightarrow \overline{X}_n \overset{P}{\rightarrow} \mu$.
\endproof

\begin{theorem}
Assume $(Z_t) \sim IID(0, \sigma^2), (\Psi_j)$ satisfies 
\[ \sum_{j = -\infty}^\infty \Psi_j< \infty, \sum_j \Psi_j \neq 0 \]
 then the stochastic process satisfies 
\[ X_t = \mu ' \sum_j \Psi_j Z_{t-j} , \]
\[ \sqrt{n}(\overline{X}_n - \mu) \overset{D}{\rightarrow} N(0, V), V = \sum_{h=-\infty}^\infty \gamma(h). \]
\end{theorem}

Approx $95\%$ confidence interval for $\mu$ is 
\[ \overline{X}_n +- 1.96 \sqrt{\frac{v}{n}}). \]
 Of course $\gamma$ is unknown, so we have to estimate it. But we cannot estimate all of them.
 We use 
\[ \overline{X}_n +- 1.96 \sqrt{\frac{\hat{v}}{n}}) , \]
 where  
\[ \hat{v} = \sum_{|h| \leq n} \frac{(n - |h|)}{n}\hat{\gamma}(h). \]


\begin{exercise}
Find the asymptotic variance $V$ for $\sqrt{n} \overline{X}_n$ in the case of $\Ar(1)$ (normal with), $(Z_t) \sim IID(0, \sigma^2)$.
\end{exercise}

Condition $\sum_h |\gamma(h)| < \infty$ is called short range dependance condition. CLT can be expanded from 
$m-$dependent and linear processes to more general stationary processes using the concept of \New{ mixing}. 

\section{Estimation of $\gamma, \rho, \alpha$ (for general linear processes)}

$\hat{\gamma}(h) = 1/n \sum_{t = 1}^{n - |h|} (X_t - \overline{X}_n)(X_{t+h}-\overline{X}_n)$

\[ \hat{\rho}(h) = \hat{\gamma}(h) / \hat{\gamma}(0) , \] 
 $\hat{\Gamma}_n , $
 as before and 
\[ \hat{R}_n = \frac{\hat{\Gamma}_n}{\hat{\gamma}(0)}. \]
 One can show all are positive semidefinite and they are consistent for linear processes.

\begin{theorem}
Assume $Z_t \sim IID(0, \sigma^2), \E[Z_t^4] < \infty$ and $\sum_j |\Psi_j| < \infty$ then the weakly stationary process 
\[ X_t = \mu ' \sum_j \Psi_j Z_{t-j} \]
satisfies
\[ \sqrt{n} \left( \begin{array}{c} \hat{\rho}_1 \\ \vdots \\ \hat{\rho}_n \end{array} \right) - 
\left( \begin{array}{c} \rho _1 \\ \vdots \\ \rho_n \end{array} \right)  \overset{D}{\rightarrow} N(0,W), \]
 where $W$ is the covariance matrix given by Bartlett formula:
\begin{align*}
 w_{i j} = \sum_{k = -\infty}^{\infty} \left[\rho(k + i)\rho(k+j) \right. \\
\left. + \rho(k-i) \rho(k+j)-2\rho(i)\rho(j)-w\rho(i)\rho(k)\rho(k+j) - 2\rho(j)\rho(k)\rho(k+i) \right] . 
\end{align*}

\end{theorem}


\begin{observation}
A similar theorem holds for $\hat{\gamma}-$s. 
\end{observation}


\begin{example}(IID case)
Assume $(X_t) \sim IID$, $EX_t^4 < \infty \Rightarrow \rho_X(h) = 0$ for $h \neq 0$. Bartlett-s formula $\Rightarrow$
$w_{ij} = \delta_{ij}$ (Kronecker's delta).

So in this case $W = I$ (identity). Asymptotically speaking 
\[ \hat{\rho}(j) \sim N(0, 1/n), j=1, \dots, h , \]
 and independent from each other! 
\end{example}

This is good, because the theorem suggests a test with null hypothesis $H_0: (X_t)$ is $ IID$.

Tests:
\begin{enumerate}
\item Visual test: approximately $95\%\; \hat{\rho}(i), i = 1, \dots, h$ should fall in 
\[  (-1.96/\sqrt{n},1.96/\sqrt{n}). \]
 If $h = 40$ there should be at most $2-3$ of these guys outside of the interval. And none should be too far from $0$.
\item Portmanteau test: $H_0: \rho(k) = k, k = 1, \dots, h$, since $\sqrt{n}\hat{\rho}(i) \approx^{iid} N(0,1), i = 1, \dots, h$. Than $Q_{n,h} = n \sum_i \hat{\rho}(i)^2 \rightarrow^d \chi^2(h)$. We reject that if $Q_{n,h} > \chi^2_{1-\alpha}(h)$.
\item Ljung Box test: $~Q_{n, h} = n \sum_i \frac{n+2}{n-i}\hat{\rho}(i)^2 \rightarrow^d \chi^2(h)$ under $H_0$, which converges faster (empirically tested)
\end{enumerate}

\begin{example} (MA(q) process)
$Z_t \sim IID(0, \sigma^2)$. Bartlett-s formula gives for $k > q$ that 
$\sqrt{n} (\hat{\rho}(k) - 0) \rightarrow N(0, 1 + 2\rho(1)^2 + \dots + 2\rho^2(q))$.

Unfortunately $\hat{\rho}(k)$ are {\bf not} asymptotically independent any more. Therefore graph of $\hat{\rho}-$s is hard to interpret.
\end{example}

Recall : if $(X_t)$ is weakly stationary with mean $0$ and $\gamma(h) \rightarrow 0$ than all $\Gamma_n$ are regular and 
$\alpha(h) = \varphi_{n,k}$, where $\Pi_{n} X_{n+1} = \varphi_{11} + \dots + \varphi_{n n}X_1$ and $\varphi -$s solve the equation, as before.
% ref
To estimate $\alpha$ we set $\hat{\alpha}(h) = \hat{\varphi}_{n n}$, the solutions of the same equations with $\Gamma, \gamma$ replaced with $\hat{\Gamma}, \hat{\gamma}$. 

Asymptotic normality of $\hat{\gamma}-$s and Slutsky imply $\alpha -$s are asymptotically normal, but the covariance matrix is even more complicated.  

\vskip1cm

More R:

library stats

library TSA $->$ good, because it does not plot $\rho(0)$

Box.test $->$ Portmanteau test. We still have to specify $h$. If $p-$value small we reject the hypothesis.
parameter type = "Ljung" $->$ Ljung test






\chapter{Arma Processes}

\begin{definition}
Weakly stationary time series $ (X:t) $ is called \New{$ \Arma(p,q)$} (auto regressive moving average) process, if it 
satisfies 
\[ X_t - \phi _1 X_{t-1} - \ldots \phi _p X_{t-p} = Z_t + \theta _1 Z_{t-1} + \ldots + \theta _q Z_{t-q}  , \label{eq:41} \]
for some real numbers $ \phi _1, \ldots , \phi _p, \theta _1, \ldots \theta _q $ and $ (Z_t) \sim WN(0, \sigma ^2) $. 
$ \phi _i$ -s and $ \theta _i $ -s are called parameters of the model.
\end{definition}

For simplicity we very often set $ (Z_t) \sim IID(0, \sigma ^2) $ .

Using backward shift operator $ B X_t = X_{t-1} $ and two polynomials $ \phi , \theta $
\begin{align*}
\phi(z) = & 1 - \phi _1 z - \ldots - \phi _p z^p \\
\theta (z) = & 1 + \theta _1 z + \ldots + \theta_q z^q \end{align*}
we can write \ref{eq:41} as 
\[ \phi (B) X_t = \theta(B) Z_t . \]

\begin{example} ($\Ma(2) $)
\[ X_t = Z_t + \theta_1 Z_{t-1} + \theta_2 Z_{t-2} , t\; t \in \Z . \]
$(Z_t) \sim \Wn(0, \sigma ^2) \Rightarrow (X_t) $ is weakly stationary and well defined.
\begin{align*} 
\phi(z) = & 1 \\
\theta (z) = & 1 + \theta _1 z + \theta_2 z^2 
\end{align*}
$ (X_t) \sim \Ma(2) = \Arma(0,2) $.
\end{example}

\begin{example} ($ \Ar(p) $)
We are looking for a stationary sequence 
\[ X_t = \phi _1 X_{t-1} + \ldots \phi _t X_{t-p} + Z_t  \]
We now have
\begin{align*}
\phi(z) = & 1 - \phi _1 z - \ldots - \phi _p z^p \\
\theta(z)  = & 1 
\end{align*}
We know that these equations do not have a stationary solution in general.

Recall : 
\[p=1, \psi _1 = \pm 1 . \]

If $ \phi _1 \in (-1, 1) $ , then stationary solution exists and it is 
\begin{align*} 
X_t = & \sum _{j=0} ^{\infty} \phi ^j Z_{t-j}  \\
\E[X_t] = & 0
\end{align*}

For $ | \phi _1 | >1 $ there exists a non-causal solution 
\[ X_t = \sum _{j=1} ^{\infty} - \phi ^{-j} Z_{t+j} . \] 


\end{example}


If real sequences $ (a_n), (b_n) $ satisfy $ \sum_j |a_j| < \infty , \sum _j |b_j| < \infty $ , 
we know that power series 
\[ \sum _{j \in \Z} a_j z^j , \; \sum _{j \in \Z} b_j z^j , \]
converge. So the same holds for 
$\xi (z) = a(z) b(z)$ 
\[ a(B) b(B) X_t = \xi (B) X_t . \]


Recall coefficients of a series representation of a function
\[ \psi (z) = \sum _j \psi _j z^j ,  \]
are uniquely defined.


\begin{theorem}
Suppose polynomials $ \phi $ and $ \theta $ have no common zeros in $ \C$, then 
\begin{itemize}
\item if \[ \phi (z) \neq 0 , \; \forall z \in \C , |z| = 1 , \]
then there exists a linear process $ (X_t) $ satisfying $ \Arma$ equation. 
\item Process $ (X_t) $, satisfying previous condition, is causal if and only if 
\[ \phi (z) = , \; \forall z \in \C, |z| \leq 1 . \label{eq:4i} \]
\end{itemize}
\end{theorem}

\begin{remark} 
Under assumption in \ref{eq:4i} linear representation of $ (X_t) $ in 
terms of $ (Z_t ) $ is determined by $ X_t = \psi (B) Z_t $, where
\[ \psi (z) = \sum_{j=- \infty} ^{\infty} \psi _j z^j = \frac{ \theta(z) }{ \phi(z)} , \]
(it absolutely converges at least on the edge of a circle in $ \C $. )
\end{remark}
\proof 
\begin{enumerate}
\item
 Since $ \phi (z) \neq 0 , \; \forall z \in \C , |z| = 1 $, then  it follows that
\[ \exists 0 < r < 1 < R < \infty ,  \; \phi (z) \neq 0 , \; \forall z \in \mathcal{R} = \{ y: r < |y| < R \} . \]
 
On $ \mathcal{R} $ function 
\[ \psi (z) = \frac{ \theta(z) }{\phi(z)}  \] 
is analytic, therefore we can write it in Laurent series 
\[ \psi (z) = \sum _{j=- \infty } ^{\infty} \psi _j z^j , \] 
which converges absolutely and uniformly on every compact subset of $ \mathcal {R} $.
It therefore follows that
\[    \sum _{j=- \infty} ^{\infty} | \psi _j | <  \infty  , \]
therefore  (lemma chapter I % ref )
\[ X_t = \psi (B) Z_t , \]
is well defined and stationary. Moreover 
\[ \phi (B) X_t  = \underbrace{ \phi(B) \psi(B) } _{ \phi(z) \cdot \psi(t) = \theta(z) } Z_t = \theta (B) Z_t . \]

\item
\begin{itemize}
\item[$\Rightarrow$] To prove this, assume $ \phi (z) \neq 0 , \forall z \in \C, |z| =  \leq 1 $, therefore it holds that $ \exists \epsilon >0 $, such that 
\[ \phi (z) \neq 0 ,  \; \forall |z| < 1 + \epsilon . \]
It then follows that 
\[ \psi (z) = \frac{ \theta (z) }{ \phi (z) } , \] 
is analytical on $ \{ z \in \C ; |z| <1 + \epsilon \} $. Therefore
\[ \psi (z) = \sum _{j=0} ^{\infty}  \psi _j z^j . \] 
This power series converges absolutely and uniformly on each compact subset of set $ \{ z \in \C, |z| < 1 + \epsilon \} $.
\[ \Rightarrow \sum _{j=0} ^{\infty} |\psi _j | < \infty \Rightarrow X_t = \sum _{j=0} ^{\infty} \psi _j Z_{t-j} , \]
is well defined and 
causal .
\item[$\Leftarrow $] Exercise.
\end{itemize}
\end{enumerate}

Recall for $ (Z_t) \sim \Wn(0, \sigma ^2) $ and $ ( \psi _j) : \sum _j |\psi _j | < \infty $.
\[ X_t = \sum _{j= - \infty }^{\infty} \psi _j Z_{t-j} , \]
is weakly station by lemma of chapter I % ref


Moreover 
\[ \gamma _X (h) = \sigma ^2 \sum _{j= - \infty } ^{\infty} \psi _j \psi _{j+|h|} , \]
and 
\[ \rho _X(h) = \sum _{j=-\infty} ^{\infty} \psi _j \psi _{j + |h| } . \]


\endproof


\begin{theorem}
Suppose $ \phi $ has a zero on the circle $ \{ z\in \C , |z | = 1 \} $, which is not zero of $ \theta $, then corresponding $ \Arma $ equation has {\bf no stationary solutions }.
\end{theorem}

\begin{example} ( $ \Ar(1) $ )
\[ X_t - \phi X_{t-1} = Z_t , \]
and 

\begin{align*}
\phi(z)  =& 1 - \phi z \\
\theta(z) =& 1
\end{align*}
$ |z| = 1 \iff | \phi |  \iff \phi = \pm 1 $.
\end{example}


\begin{example} ($\Arma(2,1) $ process)

Suppose $ \Arma $equations are
\begin{align*}
X_t - X_{t-1} + \frac{1}{4} X_{t-2} = Z_t + Z_{t-1} \\
(1 - B + \frac{1}{4} B^2 ) X_t = (1 + B) Z_t 
\end{align*}

therefore

\begin{align*}
\phi(z) = & 1 - z + \frac{1}{4} z^2 = (1 - \frac{z}{2} ) ^2 \\
\theta (z) = & 1 + z
\end{align*}
which means that $ (X_t ) $ is causal. 
If we calculate further, we get
\begin{align*} 
\psi (z) = \frac{\theta(z) } { \phi(z)} = \frac{1+z}{  1 - z + \frac{1}{4} z^2 } \\
= \sum_{j=0} ^{\infty} \psi _j z^j 
\end{align*}
We get the difference equation:
\begin{align*}
k= 0 : & \quad \psi _0 = 1 \\
k=1 : & \quad \psi _1 - \psi _0 = 2 \\
k \geq 2 : & \quad \psi _k - \psi _{k-1} + \frac{1}{4} \psi _{k-2} = 0
\end{align*}
One can show that solution to the above equation is of the form
\begin{align*}
n \geq 0 : & \quad \psi _n = 2^{-n} \left( \alpha + \beta n \right)  \\
n = 0 : & 1 = \alpha + \beta \cdot 0 \Rightarrow \alpha = 1 \\
n = 1 : & \quad 2 = \frac{1}{2} ( 1 + \beta) \Rightarrow \beta = 3 
\end{align*}
The solution is
\[ \psi _n = (1 + 3 n) \frac{1}{2^n} . \]
\end{example}


\begin{exercise}
Show that $ \Arma(1,2) $ equations 
\[  (1 - \frac{1}{2} B) X_t = (1 + \frac{1}{2} B) (1 + 0.7 B) Z_t , \]
has a causal weakly stationary solution and find coefficients in the linear representation 
\[ X_t = \sum _{j=0} ^{\infty} \psi _j z_{t-j} . \]
\end{exercise}



\begin{definition}
$\Arma(p,q) $ process $ (X_t) $ is \New{invertible} if for some sequence  $ ( \pi _j) _{j \in \N } $ satisfies
\[ \sum _{j=0} ^{\infty} | \pi _j| < \infty , \]
and 
\[ Z_t = \sum _{j=0} ^{\infty} \pi _j X_{t-j} . \]
\end{definition}

 
\begin{theorem}
Suppose $ (X_t) $  is weakly stationary $ \Arma $ process, such that $ \phi $ and $ \theta $ have
no common zeroes. Then $ (X_t ) $ is invertible $ \iff $ 
\[ \theta (z) \neq 0 , \; \forall z \in \C, |z| \leq 1 . \]
\end{theorem}

\begin{example}
\[ X_t = Z_t + \frac{1}{2} Z_{t-1} , \] 
is invertible

\[ X_t = Z_t + 1.1 Z_{t-1} , \]
is not invertible, since $ \theta (z) = 1 + 1.1 z $ has a zero in $ \{ z\in \C , |z| \leq 1 \} $ .
\end{example}



\section{ Functions $ \gamma, \rho $ and $ \alpha $ for  $ \Arma $ process }

Assume $ (X_t ) $ is causal $ \Arma $ such that 
\[ \phi (B) X_t = \theta (B) Z_t , \; t \in \Z , \]
therefore 
\[ X_t = \sum _{j=0} ^{\infty} \psi _j Z_{t-j} . \]

From lemma in chapter 1 % ref
it follows that $ \E [ X_t ] = 0 $, 
\[ \gamma (h) = \E\left[ X_t X_{t+h} \right] = \sigma ^2 \sum_ {j=0} ^{\infty} \psi _j \psi _{j+|h|} . \]

\begin{example}
\begin{enumerate}[alph]
\item
$ \Ma(q) $ process 
\begin{align*} 
 X_t = & Z_t + \theta_1  Z_{t-1} + \ldots + \theta_q Z_{t-q} \\
\gamma _X(h) = & \sigma ^2 \sum _{j=0} ^{q- |h| } \theta_j \theta_{j+ |h| } . 
\end{align*}


\item $ \Arma(1,1) $ process 
\[ X_t - \phi X_{t-1} = Z_t + \theta Z_{t-1} . \]
Suppose $ | \phi | < 1 $ , then 
\[ X_t = \sum_{j=0} ^{\infty} \psi _j Z_{t-j} . \]
\begin{align*}
 \psi (z) = \frac{ \theta(z) }{ \phi(z)} = \frac{ 1 + \theta z}{ q - \phi z} = ( 1 + \theta z) \cdot \sum _{j=0} ^{\infty} ( \phi z) ^j \\
= 1 + \sum _{k=1} ^{\infty} z^k \phi ^{k-1} (\phi + \theta) . 
\end{align*}

\begin{align*} 
\gamma _X(0) = & \sigma ^2 \cdot \sum _{j=0} ^ {\infty} \psi _j ^2 \\
= & \sigma ^2  (1 + ( \phi + \theta ) ^2 \cdot \underbrace{ \sum_{k'=0} ^{\infty} \phi ^{2k'} \big) } _{ \frac{1}{1- \phi ^2} } = 
\sigma ^2 \left( 1 + \frac{ ( \theta + \phi)^2}{1 - \phi ^2 } \right) 
\end{align*}

\begin{align*} 
\gamma(1) = \ldots = \sigma ^2 \left( (\theta + \phi ) + \frac{ ( \theta + \phi) ^2 \cdot \phi }{ 1 - \phi ^2 } \right) 
\end{align*}

\begin{align*}
\gamma(h) = \phi ^{h-1} \gamma(1) , \; h \geq 2  \; ( \searrow 0 \text{ exponentialy fast}) 
\end{align*}
For $ \sum | \gamma (h) | < \infty $ we say, that we have short range dependence!

\item $ \Ar(2) $ process 

assume 
\[  X_t - \phi _1 X_{t-1} - \phi _2 X_{t-2} = Z_t  \label{eq:ar3} , \]
$  Z_t \sim \Wn(0, \sigma^2 ) $ , we then have
\begin{align*} \phi(z) & =  1 - \phi _1 z - \phi _2 z^2 \\
 & = \left( 1 - \frac{z}{ \xi _1} \right) \left( 1 - \frac{z}{\xi_2} \right)  
\end{align*}
where $ \xi _1 , \xi _2 $ are zeros of $ \phi $, assume
\[ | \xi _1| >1 \wedge  |xi _2 | > 1 . \]
We therefore have
\[ X_t = \sum _{j=0} ^{\infty} \psi _j Z_{t-j} . \]
Moreover $  \E [X_t] = 0 $ and $ \psi _0 = 1 $ . To get 
$ \gamma _X (h) $, we multiply \ref{eq:ar3} by $ X_t , X_{t-1} \ldots  $ and calculate $ \E[ \ldots ] $ of the resulting expression.
\begin{align*} \gamma _X (0) - \phi _1 \gamma _X (1) - \phi _2 z\gamma _X(2) = \E[X_t Z_t] = \sigma ^2 \psi _0  = \sigma ^2 , \\
\gamma _X(k) - \phi _1 \gamma _X(k-1) - \phi_2 \gamma_X(k-2) = 0 . \label{eq:ar4} \end{align*}

Solution to difference equations \ref{eq:ar4} depends on characteristic polynomial:
\[ 1 - \phi _1 z - \phi _2 z^2 = 0 . \]
For simplicity assume that 
\[ \xi _1 = r e^{i \theta} , \; \xi _2 = r e^{- i \theta } , \; \theta \in [0, 2 \pi ], r \geq 1 . \]
Solution can be found as 
\[ \gamma(h) = \frac{ \sigma ^2 r ^4}{(r^2 - 1}  \frac{r^h \sin(h \theta + \psi ) } { (r^4 - 2 r^2 \cos(2 \theta + 1) \sin(\theta) } . \]
We see that $ \gamma (h) \searrow 0 $ exponentially fast. 
 \[ \phi = \arctan\left( \frac{r^2 + 1}{r^2 -1} \tan(\theta) \right) . \]

\end{enumerate}
\end{example}



{\bf PACF :} Recall that $ \Pacf $ satisfies :
\begin{align*} 
\alpha _X(1) = \rho _X(1)  \\
\alpha _X(h) = \phi _{h,h} ,
\end{align*}
where 
\[ \Gamma _h \left( \begin{array}{c}  \phi _{h1} \\ \vdots \\ \phi _{hh} \end{array} \right) = \left( \begin{array}{c} \gamma _X(1) \\ 
\vdots \\ \gamma _X(h) \end{array} \right) . \]


For $ \Ar(1) , X_t - \phi X_{t-1} = Z_t , | \phi | < 1 $ we get 
\[ \alpha _X(h) = \begin{cases} \phi ,& h =1 \\
0 ,& h \geq 2 \end{cases} . \]
\[ \Pi _h X_{h+1} = \phi X_{h} = \phi _{h1} X_h + \ldots + \phi _{hh} X_1 . \]

\begin{example}{( $ \Ar(p) $)}
$ \alpha _X(p) = \phi _p $ , $ \Pi _h X_{h+1} = \phi _1 X_h + \ldots \phi _p X_{h+1 - p} + \ldots $ and 
\[ \alpha _X(hp) = 0  . \] 
\end{example}

\begin{example} {( $ \Ma(1) $ )} 
\[ X_ t = Z_t + \theta Z_{t-1} . \]
One can show (Brockwell-Davis), that
\[ \alpha _X(h) = \phi _{h h} = \ldots = \frac{ -(- \theta ) ^{-h} }{ ( 1 + \theta ^2 + \ldots + \theta ^{2h} ) } , \]
which goes $ \searrow 0 $ exponentially fast.
\end{example}

\Important{Duality:} $ \alpha , \rho $ . By plotting $ \hat {\alpha } , \hat{\rho} $ one can get an idea if $  \Ar $ or $ \Ma $ model is suitable.

\section{ Estimation fo $ \Arma $ processes }
$ \overline{X} _n = \hat{\mu}  , \hat{\alpha}, \hat{\rho} , \hat{\gamma} $ were studied in chapter 3.

The question is how to estimate $ \phi _1, \ldots \phi _p , \theta _1, \ldots \theta _q , \sigma ^2 $ .

\New{ Yule - Walker estimators }
Suppose we want to estimate causal $ \Ar(p) $ model.
\[ X_t - \phi _1 X_{t-1} \ldots - \phi _p X_{t-p} = Z_t .  \label{eq:ar5} \]

Idea: multiply  \ref{eq:ar5} by $ X_t, X_{t-1} , \ldots $ and calculate the expectation of the resulting expression. 
\begin{align*} 
  \E[X_t ^2 ]  - \phi _1 \E[X_t X_{t-1} ] - & \ldots \phi _p \E[X_t X_{t-p} ] = \E [X_t Z_t ]  \\
\gamma_X(0) - \phi _1 \gamma _X(1) - &\ldots - \phi _p \gamma_X(p) = \sigma^2 \\
\gamma_X(1) - \phi _1 \gamma _X(0) - & \ldots - \phi _p \gamma_X(p-1) = 0 \\
& \vdots \\
\gamma_X(p) - \phi _1 \gamma _X(p-1) - & \ldots - \phi _p \gamma_X(0) = 0 ,
\end{align*}
These are \Important{Prediction equations} .

\[ \Gamma _p \left( \begin{array}{c} \phi _1 \\ \vdots \\ \phi _p \end{array} \right) = \left( \begin{array}{c} \gamma (1) \vdots \\ \ gamma (p) \end{array} \right) . \]

Note that we can write the first one of prediction equations as
\[ \gamma (0)  - \left< \left( \begin{array} {c} \phi _1 \\ \vdots \\ \phi _p \end{array} \right) , \left( \begin{array} {c} \gamma(1) \vdots \gamma(p) \end{array} \right) \right> , \]
Y-W: substitute $ \gamma $ by $ \hat{\gamma} $ to get 
\Important{ Yule Walker equations : }
\begin{align*}
 \hat{\Gamma} _p  \left( \begin{array} {c} \hat{\phi} _1 \\ \vdots \\ \hat{\phi} _p \end{array} \right) = \left( \begin{array}{c} \hat{\gamma}(1) \\ \vdots \\ \hat{\gamma}(p) \end{array} \right)  , \\
\hat{\sigma} ^2 = \hat{\gamma}(0) - \left> \left(\begin{array}{c} \hat{\phi}_1 \\ \vdots \\ \hat{\phi}_p \end{array} \right) , \left( \begin{array}{c} \hat{\gamma}(1) \\ \vdots \\ \hat{\gamma} (p) \end{array} \right) \right> 
\end{align*} 
one can show that if $ \hat{\gamma}(0) >0 \Rightarrow \hat{ \Gamma} _p $ is not singular so 
Y-W equations have a unique solution.

\begin{remark}
$ \hat{\gamma} $ are asymptotically normal and consistent, so unsurprisingly one can prove the same for $ \hat{\phi} $ .

In practice there are efficient algorithms to solve Y-W equations.
\end{remark}


\begin{example}
$ X_t = \phi X_{t-1} + Z_t, | \phi | < 1 $
\[ \hat{\gamma}(0)  \hat{\phi} = \hat{\gamma}(1· \Rightarrow \hat{\phi} = \frac{\hat{\gamma}(1)}{ \hat{\gamma}(0)} = \hat{\rho}(1) . \]
\[ \hat{ \sigma} ^2(0) = \hat{\gamma}(0)  - \frac{ \hat{\gamma}(1) \hat{\gamma}(1)}{\hat{\gamma}(0) \hat{\gamma}(0)} \hat{\gamma}(0) = \hat{\gamma}(0) ( 1 - \hat{\rho}(1)^2) \]
\end{example}


\begin{exercise}
Sho that estimators $ \hat{\sigma}^2 $ and $ \hat{\phi} $ are consistent in the previous example if 
\[ (Z_t) \sim \Iid(0, \sigma^2) . \]
\end{exercise}

One can show if $  (Z_t) \sim \Iid(0, \sigma ^2) $  then
\begin{align*} 
\sqrt{n} \left( \left( \begin{array} {c} \hat{\phi}_1 \\ \vdots \\ \hat{\phi} _p \end{array} \right) - \left( \begin{array}{c} \phi _1 \\ \vdots \\ \phi _p \end{array} \right) \right) \overset{D}{\rightarrow } \No(0, \sigma^2 \Gamma_p ^{-1} ) , \\
\hat{\sigma}^2 \overset{P}{\rightarrow} \sigma^2  .
\end{align*}

Therefore one can construct confidence intervals.


\section{Gaussian Maximu Likelihood}

We want to estimate $ \theta _1, \ldots , \theta _q, \phi _1, \ldots , \phi _p, \sigma z^2 $ in $ \Arma(p,q) $.

\[ X_t = \sum \psi _j Z_{t-j} . \]
Simplifying assumption:
\begin{align*}
 ((X_t) \text{ is gaussian, invertible and causal} \\
(Z_t) \overset{I.I.D.}{\sim} \No(0, \sigma^2) .
\end{align*}

For a vector of observations and parameters 
\begin{align*}
\vec{X}_t  = \left( \begin{array}{c} X_1  \\ \vdots \\ X_t \end{array} \right) , \quad
\vec{\phi} = \left( \begin{array}{c} \phi _1 \\ \vdots \\ \phi _p \end{array} \right) , \quad
\vec{\theta } =  \left( \begin{array}{c} \theta _1 \\ \vdots \\ \theta _q \end{array} \right)  .
\end{align*}

We maximize likelihood with respect to $ (\vec{\phi}, \vec{\theta}, \sigma ^2 ) $ :
\begin{align*}
 L(\vec{\phi}, \vec{\theta}, \sigma ^2 & =  L( \Gamma _t ( \vec{\phi}, \vec{\theta}, \sigma z^2)) \\
 & = (2 \pi)^{- \frac{t}{2}} \frac{1}{ \sqrt{|\Gamma_t| }} e^{- \frac{1}{2} \vec{X}^T _t \Gamma _t ^{-1} X_t  } .
\end{align*}

to find $ \hat{\vec{\phi}} , \hat{\vec{\theta}} , \hat{\sigma}^2 $, we use  % Missing Word
optimization.

this works even if data are not gaussian (Brockwell-Davies).

One can show that for casually invertible gaussian $ \Arma $ process with $ \Iid $ noise 

\[ 
\sqrt{t} \left[
\left( \begin{array}{c} \hat{\phi} _1 \\ \vdots \\  \hat{\phi} _p \\ \hat{\theta} _1 \\ \vdots \\ \hat{\theta} _q \end{array} \right)
- \left( \begin{array}{c} \phi _1 \\ \vdots \\ \phi _p \\ \theta _1 \\ \vdots \\ \theta _q \end{array} \right) \right] 
\overset{D}{\rightarrow} \No(0, W ) , \]
where $ W $ is something very complex.

If $( X_t) $ is weakly stationary with mean zero, then $ (X_t) $ is 
essentially a causal linear process and a deterministic sequence.

\begin{align*}
 M_n = \Span(X_i : i = - \infty, \ldots , n ) \\
 \sigma ^2 = \E[ X_{n+1} - \Pi_{M_n} X_{n+1})^2 ] 
\end{align*}

\New{Wold decomposition} theorem says that if $ \sigma ^2 > 0 $, then
$ X_t = \sum _{j=0} ^{\infty} \psi _j Z_{t-j} + V-t $ where $ (Z_t ) \sim \Wn(0, \sigma ^2) $ and 
$ (V_t) $ is deterministic sequence.




\section{Model selection}

How to choose $ p, $ an $ q $ =

\New{Akaike information criterion}

\begin{enumerate}
\item $ \Aic $
\[ \Aic = - 2 (\max) \Loglikelihood +  \underbrace{2 \cdot K}_{ \text{ penalty} } , \]
where $ K = (p+q + 1) = \#\text{ of parameters} $, (penalty = you pay $ 2 \epsilon $ for every extra parameter).

Select $ p, q $ , which give minimal $ \Aic $ (has tendency to overfit).

\item  $ \Aicc $ (corrected $\Aic $) 
\[ \Aicc = - 2(\max) \Loglikelihood + 2K \cdot \frac{n}{n - k + 1} , \]
$ n = \text{sample size} $.

\item \New{Bayesian information criteria} ( $ \Bic$ )

\[ \Bic = 2 (\max)\Loglikelihood + K \cdot \log(n) , \]
this is consistent criterion ( $ \Aic $ is only asymptotically efficient).

\end{enumerate}

\begin{remark}
Software packages in R procedures, books differ in 
definition of this criteria.

{\bf R : }

ar has minimal $ \Aic $ always equal to $ 0 $., looking at residuals after
fitting allows us to detect goodness of fit.
\begin{itemize}
\item[ \it ar ] selects $ p, q $.
\item[ \it arma and arima ] you have to give $ p, q $
\item[\it qqnorm] compares data to normal distribution.
\item[ \it arima plot ] we can use diag which plots residuals over time
\end{itemize}

\end{remark}


\section{ Forecasting $\Arma $ models }

We want to find best linear predictor for $ X_{n+1} $ in terms of
$ X_1, X_2, \ldots , X_n : $
\[ X_n+1 = \phi_1 X_n +  \ldots \phi _n X_1 , \]
\Important{ we know $ \phi _{n-1,1}, \ldots , \phi_{n-1, n-1} $} .
Prediction equations
\[ \Gamma _n \vec{\phi} _n = \left( \begin{array}{c} \gamma(1) \\ \vdots \\ \gamma(n) \end{array} \right), \]
where
\[ \vec{\phi}_n = \left( \begin{array}{c} \phi _1 \\ \vdots \\ \phi _n \end{array} \right) . \]

\New{ Mean Squared Error} 
\[ \E \left[ (X_{n+1} - \Pi _n X_{n+1} )^2  \right] = \gamma_X(0) - \left< \vec{\phi }_n ,  \left( \begin{array}{c} \gamma(1) \\ \vdots \\
\gamma(n) \end{array} \right) \right> := v_n . \]

Recall $ \gamma (0) >0 , \gamma(h) \to 0  \; \Rightarrow \Gamma _n $ are 
regular $ (X_t) $ is stationary, with mean zero and stationary. If this holds
 $ \phi _i $ -s can be calculated recursively.

\New{ Dorbin - Levinson algorithm}
The coefficient $ \phi _{n,j}, j=1, \ldots , n, n=1, \ldots ,$ can be computed as follows
\[ v_0 = \gamma(0) , \]
for n in $ 1:N $ 
\begin{align*}
\left( \begin{array}{c} \phi _{n,1} \vdots \\ \phi _{n, n} \end{array} \right) = 
\left( \begin{array}{c} \phi _{n-1, 1} \\ \vdots\\ \phi _{n-1, n-1} \end{array} \right) - 
\phi _{n,n} \left( \begin{array}{c} \phi_{n-1, n-1} \\ \vdots \\ \phi _{n-1, 1} \end{array} \right) \\
v_n = v_{n-1} (1 - \phi _{n n})^2  
.
\end{align*}

Dorbin-Levinson algorithm is used to get
\begin{itemize}
\item  best linear predictors,
\item $ \Pacf $ ,
\item find Yule-Walker estimators.
\end{itemize}


For general $\Arma(p,q) $ processes, it is more reasonable to predict $ X_{n+1} $ in terms of \New{innovations}.
\begin{align*}
\hat{X} _1 = 1, \hat{X}_n = \Pi_{n-1} X_n : \; X_n - \hat{X} _n , \; n =1 , \ldots, N.  \\
\span\{X_1 - \hat{X}_1 , \ldots , X_n - \hat{X}_n \} = \Span\{X_1, \ldots , X_n \} , 
\end{align*}
but innovations are mutually orthogonal and we can write
\[ \hat{X} _{n+1} = \Pi _n X_{n+1} = \theta _{n,1} (X_n - \hat{X} _n) + \ldots + \theta_{n,n} (X_1 - \hat{X}_1 ). \]
One can find $ \theta _{n,j} $ -s recursively.

\New{Innovations algorithm}

Assume $ (X_t) $ is weakly stationary, with mean zero, such that  \ref{eq:star} holds. Then coefficient 
$ \theta _{n,j} $ can be computed as follows
\[ v_0 = \gamma(0), \]
for $ n=1:N $ .
\begin{align*}
 \theta_{n, n-k} = \frac{1}{v_k} \left( \gamma(n-k) - \sum _{j=0} ^{k-1} \theta_{k,k-j} \theta_{n, n-j} \cdot v_j \right) \\
v_n = \gamma(0) - \sum _{j=0} ^{n-1} \theta_{n,n-j} ^2 v_j .
\end{align*}


Assume $ (X_t) $ is causal $ \Arma(p,q) $  such that:
\begin{align*}
\phi (B) X_t = \theta(B) Z_t , \; Z_t \sim \Wn(0, \sigma ^2) \\
\hat{X}_{n+1} = \Pi _n X_{n+1} , \\
v_n = \E\left[ \left( X_{n+1} - \hat{X} _{n+1} \right)^2 \right] = - \sigma ^2 \sqrt{n}  \\
X_t = \phi _1 X_{t-1} - \ldots  - \phi _p X_{t-p} + Z_t + \ldots + Z_t +  \theta _1 X_{t-1} + \ldots + \theta _q Z_{t-q }
\end{align*}

\New{Linear prediction of  $\Arma(p,q) $ processes}

\begin{proposition}
For causal $ \Arma(p,q) $ and $ m = \max(p,q) $.
\begin{align*} \hat{X}_{n+1} \\ = 
 \begin{cases} 
\sum_{j=1}^n \theta_{n,j} (X_{n+1-j} - \hat{X}_{n+1-j} ) , & q \leq n <m \\
\phi X_n + \ldots + \phi _p X_{n+1-p} + \sum _{j=1}^q \theta_{n,j}\left( x_{n+1-j} - \hat{X}_{n+1-j} \right) , & n \geq m 
\end{cases}
\end{align*}

If $ (X_t) $ is also invertible
 \begin{align*}
\theta_{n,j} \to \theta _j, & \text{ as } n \to \infty \\
\sqrt{n} \to 1 , & \text{ as } n \to \infty 
\end{align*}
\label{prop:6}
\end{proposition}

Proposition \ref{prop6} says $ \hat{X} _{n+1} $ can be calculated by rewriting $ \Arma $ equations using 
innovations instead of the noise
and prediction error is asymptotically optimal.

If $ p= 0 $  $(X_t) \sim \Ma(q) , $ innovation algorithm can be used to
obtain estimates of  $ \theta_j$ -s (of course using $ \hat{\gamma} $ instead of $ \gamma $ ),
thus asymptotically normal under technical conditions!
( $ g_n \to \infty , g_n/n \to 0 $)


\section{ Prediction intervals}

Assume  $ X_1, X_2, z\ldots X_n, X_{n+1}, \ldots , X_{n+h} $ is gaussian vector with mean zero.
\[ \Pi_n X_{n+h} = \E[ X_{n+h} | X_1, \ldots , X_n ] , \]
best linear predictor is also the best predictor.

Prediction
\[ X_{n+h} - \Pi _n X_{n+h} , \]
 
is also $ \No(0, \sigma^2(h)) $,
\[ h=1, \; v_n = \sigma^2(1) . \]

$(1- \alpha) $ prediction interval
\[ \Pi _n X_{n+h} \pm Z_{\alpha/2} \sigma_n(h)  . \]

In practice we do not know $ \gamma $ -s or $ \phi $ -s or $ \theta $ -s or $ \theta_{n,j} $ -s,
so they have to be estimated.

Forecast is obtained if substitute these values by $ \hat{\gamma}, \hat{\phi}, \hat{\theta}$ -s.
Then prediction interval does not take into account
\begin{itemize}
\item parameter of uncertainty,
\item model uncertainty,
\item gaussianity might not hold,
\item there is also uncertainty about trend and seasonality,
\item prediction errors are controlled only pointwise.
\end{itemize}

 
\section{ Nonstationary models}

For $ d \in \N _0 , (X_t) $ is called  \New{$ \Arima(p,d, q) $ process} if 
\[ Y_t = (1-B)^{d} X_t , \]
where $ (X_t) $ is causal $ \Arma(p,q) $ process.
Thus it satisfies
\[ \phi(B) (1- B)^d X_t = \theta(B) Z_t , \]
where $ Z_t \sim \Wn(0,  \sigma^2) $, $ \phi, \theta $ polynomials and $ \phi (z) \neq 0 $ for 
$ |z| \leq 1 $.
\[ X_t  \text{ is stationary } \iff d = 0! \]

For $ d=1 , Y_t = X_t - X_{t-1} $.
Therefore 
\[ X_t = X_0 + Y_1 + Y_2 + \ldots + Y_t ,  \; Y_t \sim \Arma(p,q)  , \]

e.g. $ \Arima(1,1, 0) $, then 
\[ Y_t = \phi Y_{t-1} + Z_t = \sum _{j=0}^{\infty} \phi ^j Z_{t-j} . \]

$ \Arima $ processes (for $ d \geq 1 $ ) are characterized by slow
decay of acf.

Note polynomial $ \phi(z) (1- z)^d $ has root $ z= 1 $, so one can
test for this using so called unit root tests.

There are $ \Sarima $ (seas.) , $ \Farima $ (long range dependent),
$ \Inar, \ldots $.


\chapter{Nonlinear models}

\begin{itemize}
\item In finance: econometrics
\item in climate: 
\[ X_t = X_0 e^{X_1 + \ldots + X_t } . \]
\item integer valued time series.
\end{itemize}

\New{ $ \Garch $ and related }

\[ X_t = \log \frac{S_t}{S_{t-1}} , \]
there are "stylized facts".
\begin{enumerate}
\item $ \Acf $ is approximately $ 0 $ at all lags $ h \geq 1 $
\item $ \Acf $ of $ |X_t | $ of $ X_t ^2 $ decays slowly.
\item extremes are very large (with respect to Gaussian distrib.) and come in clusters.
\end{enumerate}

$ \Ar(p) , \Ma(q) $ cannot model this.

\New{ $ \Arch $ } model ( Engle 1982 )
\[ Z_t \sim \Iid(0, 1) , \]
typically  $ \No(0,1) $. $ \alpha _0, \alpha _1 > 0 $. We look for stationary solution $ \neq 0 $ 
\begin{align*}
X_t = \sigma _t Z_t \\
\text{ volatility } \sigma _t ^2 = \alpha _0 + \alpha_1 X_{t-1}^2 
.
\end{align*}

Stationary solution exists for other $ \alpha _1 $ -s. They solve the above equations. 
This solution is strongly stationary and \Important{nonanticipativew} i.e.
\[ \sigma_t \in \sigma\{ Z_{t-1}, Z_{t-2}, \ldots \} . \]
\[ X_t = \sigma_t Z_t \in \sigma \{ Z_t , Z_{t-1}, Z_{t-2}, \ldots \} , \]
so $ X_t $ is independent of $ \sigma \{ Z_{t+1}, Z_{t+2}, \ldots \}  . $. 

Assume $ ( \sigma _t, X_t ) $ is nonanticipative solution of equations \ref{eq:51}., 
if $ \E[\sigma _t^2] < \infty \Rightarrow \E[X_t]  = \E[ \sigma_t Z_t ] = 0 . $
( $ \E[ \sigma_t^2 ] = \alpha _0 + \alpha_1 \E[   \sigma_t^2] $).
( $h \geq 1 $)
\[ \gamma _X(h) = \E[X_t X_{t+h} ] = \E[ \underbrace{X_t \sigma_{t+h} Z_{t+h} } _{\text{independence }}] = 0 , \]

\[ \E[X_t^2 | X_{t-1}, X_{t-2}, \ldots ] = \E[ \sigma_t^2 Z_t^2 | X_{t-1}, \ldots] = \sigma_t^2 \E[Z_t^2] = \sigma_t^2 , \]
since $ Z_t, X_{t-j} $ are independent ... conditional heteroscedastic.

$ \Arch(p) $ model
\[ Z_t \sim \Iid(0, 1) , \]
typically $ \No(0,1) $. 
$  \alpha_0 , \alpha _1, \ldots , \alpha_p > 0 $.

Look for stationary solution $ \neq 0 $.
\begin{align*}
 X_t = \sigma_t Z_t , \\
\sigma_t^2 = \alpha _0 + \alpha_1 X_{t-1}^2 + \ldots + \alpha_p X_{t-1} ^2 , \\
\phi(z) = 1 - \sum_{i=1}^p \alpha_i z^i .
\end{align*}
Therefore we get
\[ \phi(B) X_t^2 = \underbrace{ \left( Z_t^2 -1\right) \sigma_t^2 }_{\nu_t} + \alpha_0 . \]

If $( \sigma _t, X_t) $ are stationary solution  and nonanticipative of \ref{eq:5w} and
if 
\[  \E[ \sigma_t^4] < \infty, \; \E[Z^4] < \infty \Rightarrow \nu _t , \]
are white noise.
\begin{align*}
\E[ \nu_t] = \E\left[ \left( Z_t^2 - 1 \right) \sigma_t^2 \right] = 0 \\
\E[ \nu_t \nu_{t+h} ] = 0,
\end{align*}
similarly for large $ p$.

$\Garch(p,q) $ (Boleslev 1986)

\[ Z_t \sim \Iid(0,1) , \]
$ \alpha_0, \alpha_p, \beta_q >0 $ and $ \alpha_1, \ldots , \alpha_{p-1}, \beta_1, \ldots , \beta_{q-1} \geq 0 $. 
We look for stationary solution $ \neq 0 $
\begin{align*}
X_t = \sigma_t Z_t \\
\sigma_t ^2 = \alpha_0 + \sum _{i=1}^p \alpha_i X_{t-i}^2 + \sum_{j=1}^q \beta_j \sigma_{t-j}^2 .
\end{align*}

\begin{remark}
$ \Garch(p,q) $ provides reasonable model for relatively small $ p $ and $ q $.
There exists estimation and prediction theory, which is related to $ \Arma $.
\end{remark}

{\bf Questions: }
\begin{itemize}
\item when do we have stationary solution to \ref{eq:53}. ?
\item can you describe its distribution ?
\item how to estimate and predict?
\end{itemize}

\begin{example} $ \Garch(1,1) $ 
\[ \sigma_t^2 = \alpha_0 + \alpha_1 X_{t-1}^2 + \beta_1 \sigma_{t-1}^2  \\
= \alpha_0 + \sigma_{t-1}^2 \underbrace{(\alpha_1 Z_{t-1}^2 + \beta_1) }_{ A_t} . \]
(stochastic recurrence equation)

\begin{align*}
 X_t ^2 - \sigma_t^2 = \sigma_t ^2( Z_t^2 - 1) = \nu_t \\
- \beta_1(X_{t-1}^2 - \sigma_{t-1}^2) = \beta_1 \nu_{t-1} \\
X_t^2 - \beta_1 X_{t-1}^2 -  \underbrace{ ( \sigma_t^2 - \beta_1 \sigma_{t-1}^2)}_{ \alpha_0 + \alpha_1 X_{t-1}^2} = \nu_t - \beta_1 + \nu_{t-1} 
\end{align*}


\[ X_t^2 - (\alpha_1 + \beta_1) X_{t-1}^2 = \alpha_0 + \underbrace{ \nu_t - \beta_1 \nu_{t-1} }_{\Ma() \text{ process}} , \]
if $ \E[\sigma_t^2], \E[Z^4] < \infty \Rightarrow \nu _t $ is white noise.
\end{example}

\section{Gaussian quasi-max likelihood }

Assume $ X_1, \ldots , X_n $ form a stationary $ \Garch(p,q) $ model. Density of the data
\begin{align*}
 f_{X_1, \ldots, X_n}(x_1, \ldots , x_n) =  \underbrace{ f_{X_1, \ldots , X_p}(x_1, \ldots , x_p)}_{\text{ we forget this }  } \\
\underbrace{\prod_{t=p+1}^n f_{X|X_1, \ldots , X_{t-1}}(x_t | x_1, \ldots , x_{t-1} ) }_{\text{we optimize this}} \label{eq:54} , \end{align*}
assume $ Z_t \overset{i.i.d.}{\sim} \No(0, 1) $ all conditional densities in \ref{eq:54} are
normal.

\begin{remark}
These estimators are asymptotically normal with rate $ \sqrt{n} $. (using numerical optimization)

More realistic assumption on $ Z_i$ -s can give non-consistent estimators!
\end{remark}



\section{ Stochastic Volatility Models}
Let $ Z_t \sim \Iid $ , with distribution function $ F $.

$ (\sigma_t^2) $ nonnegative stationary process independent of $ (Z_t) $.
\[ X_t = \sigma_t Z_t . \]

Typical choice is 
\[
\sigma_t = e^{Y_t} , \; Y_t = \sum_{j=0}^{\infty} \psi _j \nu _{t-j}  \; \text{ (causal linear process) } ,
\]
$ \eta _t \sim \Iid $, if $ \eta \overset{iid}{\sim} \No(0, \sigma^2) , \E[ \psi^2] < \infty \Rightarrow Y_t $ are normal
\begin{align*}
\eta_t \sim \Iid, \\
\end{align*}
therefore $ \sigma_t $ are lognormal and $ (X_t) $ is well defined.

In both families ( $ \Garch $ and $ \Sv $) we can also add
nonzero mean.
\[ X_t' = X_t + \mu, \; \mu \in \R , \]
where $ (X_t) $ is $ \Garch $ or $ \Sv $ process.


\section{Stochastic recurrence equation}

$ \Garch(1,1) $ can be written in terms of $ S.R.E $ i.e. in the form 
\[ Y_t = A_t Y_{t-1} + B_t , \]
$ (A_t, B_t) \overset{iid}{\sim} $ random vectors.

Question: when \ref{eq:55} has a stationary solution. Iterate backward
\[ Y_t = A_t \cdot \ldots \cdot A_{t-k} Y_{t-k-1} + \sum _{i=t-k}^t A_t \cdot \ldots \cdot A_{it+1}B_i . \]
We guess that the solution is of the form:
\[ Y_t = \sum _{i = - \infty}^{\infty} A_t \cdot \ldots \cdot A_{i + 1} B_i  . \label{eq:56} \]

If \ref{eq:56} converges a.s. it satisfies eq. \ref{eq:55} moreover and it is 
non-anticipative! 


\begin{theorem} Assume $ (A_t, B_t) $ form an iid sequence, such that $ A_t, B_t > 0,$
\begin{align*} 
- \infty \leq \E[ \log(A)] < 0 \\
\E[ |\log B_1|] < \infty .
\end{align*}
Then series in \ref{eq:56} converges a.s. and represents a stationary solution to \ref{eq:55} .
\end{theorem}
\proof HW, 
\\
Idea: $ Y_t = \sum_{i = - \infty}^t A_t \cdot \ldots \cdot A_{i+1} B_i . $ % Add (there is more hing than that)
\endproof


\begin{example}  $ \Arch(1) $

\[ \sigma_t ^2 = \alpha_0 + \alpha_1 Z_{t-1}^2 \sigma_{t-1}^2 .\]

\begin{align*}
 \E[ \log A] = 0 \equiv \E[\log(\alpha_1 Z_t^2)] < \infty \\
\equiv \alpha_1 < e^{ - \E[\log Z_1^2] }  ,
\end{align*}
If $ Z_i \overset{iid}{\sim} \No(0,1) $
\[ - \E[\log Z_1^2] = \gamma + \log 2 , \]
where $ \gamma $ is Euler's constant.

So $ \alpha _1 < 2 e^{\gamma} \approx 3.56 $, so $ \alpha_1 > 3.56 $, there is no stationary $ \Arch $ with normal noise!

\end{example}


\begin{example} $ \Garch(1,1) $

\[ \sigma^2 = \underbrace{ (\alpha_1 Z_{t-1}^2 + \beta_1)}_{A_t}  \sigma_{t-1}^2 \alpha_0 , \]
sufficient but not needed. 

\[ \E \log(\alpha_1 Z_t^2 + \beta_1)]  \overset{Jense}{\leq} \log(\alpha_1 + \beta_1 < 0 . \]


\section{ Properties of Stationary Solution}

For $ \Garch(1,1) $ , if $ \alpha _1 + \beta _1 <1 $
\begin{align*}
 \sigma _t ^2 = \alpha_0 + \alpha_0 \sum_{i=1}^{\infty}(\alpha_1 + \beta_1)^j < \infty \\
= \frac{\alpha_0}{1 - (\alpha_1 + \beta_1)} < \infty . 
\end{align*}
therefore $ \E[X_t^2 ] = \E[\sigma_t^2 Z_t^2] = \frac{\alpha_0}{1- (\alpha_1 + \beta_1)} < \infty . $
therefore $ (X_t) $ is weakly and strongly stationary if $ \alpha_1 + \beta_1 < 1 $.

Solution $ \alpha_1 + \beta_1 \geq 1 $ but then we can't have weak stationarity.

\section{Tails Of Stationary Solutions To S.R.E}

\[ Y_t = A_t Y_{t-1} + B_t, \]
$(A_t, B_t) $ i.i.d. Assume $ A_t, B_t > 0 $ and $ A_t $ have density 
\[ f(t) \sim g(t) , \; \frac{f(t)}{g(t)} \overset{t \to \infty}{\rightarrow } 1 . \]
\end{example}


\begin{theorem}
Assume $ \mathcal{K} > 0 , \epsilon > 0 $ such that 
\[ \E[A^{\mathcal{K}}] =1, \quad \E[ B^{\mathcal{K} \epsilon} ] < \infty, \quad \E[ A^{\mathcal{K}} \log A] < \infty \]
$ \Rightarrow $  then \ref{eq:55}  has stationary solution and 
\[ P(Y> u) \sim c u^{- \mathcal{K}} , \]
for $ u \to \infty $ power - law tail.
\end{theorem}

For $ Z \sim \No(0,1) $ :
\[ P(Z > u) \sim \frac{1}{u} e^{-u^2 /2} , \]
$ \searrow 0 $ supercop.fast.

Then $ 2 $ applied on $ P( \sigma_t^2  > u \sim c u ^{-\mathcal{K}} $.
\[ X_t = \sigma_t Z_t, \]
$ Z_t  $ is symmetric, where $ \E[ (\alpha_1 Z^2 + \beta_1)^{\mathcal{K}} ] =1 $.
\[ \Rightarrow \ldots \Rightarrow P(X_t >u) \sim \frac{1}{2} c u ^{-2 \mathcal{K}} \]
$ (Z) \Rightarrow \E[ Y^{\mathcal{K} + \delta} ] =  + \infty , \forall \delta > 0 $
If $ \mathcal{K}  < 2 $ we don't have variance.

If $ \mathcal{K} < 1 $ then $ (Y_t) $ do not have expectation.

\section{ Prediction for $ \Garch $ models}

If we know distribution of $ Z_t $ -s
\[ \sigma_{t+1}^2 = \alpha_0 + \alpha_1 X_{t-1}^2 + \ldots + \beta_q \sigma_{t+q}^2 . \]
$ \Rightarrow X_t | X_{t-1} $ has known distribution ( $ X_t  = \sigma_t Z_t $)

For $ Z_t \sim N(0,1) $ , if we estimate $ \hat{\alpha}_0, \hat{\alpha}_1, \ldots, \hat{\beta}_q $ prediction interval
for $ X_{t+1} $ is\[ \pm 1.96 \hat{\sigma}_{t+1} , \\
\hat{\sigma}_{t+1}^2 = \hat{\alpha}_0 + \hat{\alpha}_1 X_{t-1}^2 + \ldots + \hat{\beta}_q \hat{\sigma}_{t+1-q}^2 . \]


library(fGarch)
instead in library tsa use garch

residuals aren't normal, then prediction interval for $ X_{t+1} $ should be 
\[ \pm q \_{\alpha/2} \hat{\sigma}_{t+1} \]
Instead of $ 1.96$.












\end{document} 

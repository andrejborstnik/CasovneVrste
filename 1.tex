\documentclass[12pt,a4paper]{amsart}
% ukazi za delo s slovenscino -- izberi kodiranje, ki ti ustreza
\usepackage[slovene]{babel}
%\usepackage[cp12500]{inputenc}
%\usepackage[T1]{fontenc}
\usepackage[utf8]{inputenc}
\usepackage{amsmath,amssymb,amsfonts}
\usepackage{url}
\usepackage{graphicx}
\graphicspath{{./Slike/}}
%\usepackage[normalem]{ulem}
\usepackage[dvipsnames,usenames]{color}

\setlength{\evensidemargin}{-0mm}
\setlength{\oddsidemargin}{-0mm}
\setlength{\headheight}{0mm}
\setlength{\headsep}{0mm}
\setlength{\textheight}{260mm}
\setlength{\textwidth}{160mm}
\setlength{\footskip}{10mm}
\setlength{\topmargin}{-10mm}

\textwidth 15cm
\textheight 24cm
\oddsidemargin.5cm
\evensidemargin.5cm
\topmargin-5mm
\addtolength{\footskip}{10pt}
\pagestyle{plain}
\overfullrule=15pt % oznaci predlogo vrstico
\setlength{\parindent}{0pt}


% ukazi za matematicna okolja
\theoremstyle{definition} % tekst napisan pokoncno
\newtheorem{defn}{Definition}[section]
\newtheorem{example}[defn]{Example}
\newtheorem{excercise}[defn]{excercise}
\newtheorem{observation}[defn]{Observation}

\theoremstyle{plain} % tekst napisan posevno
\newtheorem{lema}[defn]{Lema}
\newtheorem{theorem}[defn]{Theorem}
\newtheorem{claim}[defn]{Claim}
\newtheorem{collary}[defn]{Collary}


% za stevilske mnozice uporabi naslednje simbole
\newcommand{\R}{\mathbb R}
\newcommand{\N}{\mathbb N}
\newcommand{\Z}{\mathbb Z}
\newcommand{\C}{\mathbb C}
\newcommand{\Q}{\mathbb Q}
\newcommand{\F}{\mathbb F}


\begin{document}

{\bf \huge Time series}

\vskip1cm

{\bf Predavatelj : } Bojan Basrak

% https://web.math.pmf.unizg.hr/~bbasrak/
% www.math.hr/~bbasrak
% {\bf www: } https://web.math.pmf.unizg.hr/$\tilde$bbasrak/


\section{Basic models and terminology}

Stohastic process $(X_i)_{i \in I}$ on the same probability sapace.

Time series is a sequence of observation, collected over time - prices of stock, level of the sea, ... $ =  (x_i)_{i = 1, \dots, n}$
Ideally at equidestant time points. 

It is also a stohastic procces indexed in discrete time, i.e.  $I = \{1, \dots, n\}, \N, \Z$ (model). Typically $I \subset \Z$.
We will be looking at final dimentional distributions of $(X_t)_{t \in \Z}$: family of distributions of a random vector
$(X_{t_1}, \dots, X_{t_k})$, $t_1 < \dots < t_k$.
They determine the distribution of the whole process $(X_t)_{t \in \Z}$.
$(X_t)_{t \in \Z}: \omega \rightarrow \R^{\Z} $.

Examples:

\begin{itemize}
\item $X_i \tilde (i.i.d.) F$ -> $(X_i)$ is a time series (model). But is independant of time -> bad
\item $X_i \tilde F$, $S_0 = 0$, random walk $S_n = S_{n - 1} + X_n, n\geq 1$. Not interesting, because the model is changing constantly. 
\end{itemize}

\begin{defn}
For a time series $(X_t){t\in \Z}$, satisfying $Var(X_t) < <\infty$ we define {\bf the autocovarinace function} $\gamma_X : \Z \times\Z \rightarrow  \R$
$\gamma_X(s,y) = Cov(X_s, X_t) = E[(X_s - EX_s)(X_t - EX_t)]$
\end{defn}

\begin{defn}
Time series $(X_t)_t$ is weakly stationary (če nič ne rečemo je stationary = weakly stationary) if 
\begin{itemize}
\item $E|X_t|^2 < \infty$, for all $t$
\item $EX_t = m$ (const), for all $t$
\item $Cov(X_s, X_t) = Cov(X_{s+h},X_{t+h})$, $h, s, t \in \Z$
\end{itemize}
Covariance represents dependance of the variables. 
\end{defn}

\begin{observation}
For weakly stationary time series the function $\gamma_X$ is actually a function of one variable only. 
$\gamma_X : \Z \rightarrow \R$,
$\gamma_X(h) = Cov(X_h, X_0)$.

This is a good definition, because $Cov(X_s, X_t) = Cov(X_{s-t},X_0) = \gamma_X(s-t)$. 
\end{observation}


\begin{defn}
The autocorrelation function of weakly stationary time series $(X_t)$ is a function
$\gamma_X : \Z \rightarrow [-1, 1]$
$\rho_X(h) = \frac{\gamma_X(h)}{\gamma_X(0)} = Corr(X_{t+h},X_t)$
(we assume $\gamma_X(0) > 0$). If it is $0$ everything is a constant and we have no interest in observing it.
\end{defn}

\begin{defn}
Time series $(X_t)$ is strictly (strongly) stationary if the distribution of a random vector $(X_{t_1 + h}, \dots X_{t_k + h})$ does not depend on $h$ for every $k\in \N, t_1, \dots, t_k$.
\end{defn}

\begin{excercise}
Strict stationarity + $EX^2 < \infty$ -> weak stationarity.
\end{excercise}

For a given time series we define a family 
$\{  F_{\bf t} \}_{\bf t \in T}$, where $T = \{(t_1, \dots, t_k), t_1 < \dots < t_k, k \in \N\}$.
$F_{\bf t}({\bf X}) = P(X_t \leq X-1, \dots, X_{t_k \leq x_k}), {\bf x} = (x_1 \dots, x_k), {\bf t} = (t_1, \dots t_k)$
$F_{ \bf t}$ determines the distribution of the time series


\begin{theorem}
The family $\{F_{\bf t}\}_{\bf t \in T}$ is a class of f.d.d (final dimention distribution) function of a time series
$\Leftrightarrow$
for each $k, {\bf t } \in T: \lim_{x_i \rightarrow \infty} F_{\bf t}({\bf x}) = F_{\bf t(i)}({\bf x(i)})$
for every ${\bf x(i)} \in R^{k-1} = (x_1,\dots,x_{k-1},x_{k+1},\dots, x_k)$ and the same for ${\bf t}$ .
\end{theorem}
\proof
No proof, very tehnical. 
$\rightarrow$ is very easy to prove. Ti is enough to observe the probability is continous with respect to groving events.
\endproof



\begin{defn}
Random vector ${\bf Y } = (Y_1, \dots, Y_k)^T$ has a multivariate normal (gaussian) distribution if there exists a vector $a \in \R^k$ and a matrix $B \in M_{k\times m}$ and a random vector ${\bf X} = (X_1, \dots, X_m)^T$ such that $X_i \tilde(iid) N(0,1)$, such that ${\bf Y} = a + B {\bf X}$.

In this case we know that $E{\bf Y} = (EY_1, \dots, EY_k)^T = (a_1, \dots, a_k)$ and the covariance matrix of ${\bf Y}$ is given by $\sum_y = B B^T$.
We write $Y \tilde N(a, \sum_y)$.

\end{defn}

% todo popravi
If $det (\sum_y) > 0$, ${\bf Y}$ has density and $f_{\bf Y}(y) = (2 \pi)^{-k/2}\frac{1}{\sqrt{det(\sum_y)}} e^{-1/2<{\bf Y} - a, \sum^{-1}({\bf y} - a)>}$.

\begin{defn}
A random proces $(X_t)_t$ is gaussian (normal) if its f.d.d.-s are all multivariate normal.
\end{defn} 


\begin{excercise}
Assume $(X_t)_t$ is a gaussian time series with constant expectation. Then the autocovariance function $\gamma_X$ determines all f.d.d-s.

If we look at the vector $(X_, \dots)$ we see it is distributed normally. We can read the covariance matrix form $\gamma_X$, and this is all we need to show.
\end{excercise}

\begin{excercise}
Assume $(X_t)_t$ is a gaussian time series and weakly stationary. Then it is strictly stationry as well. 
\end{excercise}


\begin{example}
A weakly stationary time series $(X_t)$ is called {\bf white noise } if $EX_t = 0$ for all $t$ and 
$$\gamma_X(h) =  \sigma^2 = Var(X_t) ,h = 0
			0, h \neq 0
$$
We write $(X_t) \tilde WN(0, \sigma^2)$.

Clearly if $(X_t)$ i.i.d. and $EX_t = 0$ and $Var X_t < \infty \Rightarrow (X_t) \tilde WN(0, \sigma^2)$.  
\end{example}


\begin{excercise}
Find an example of a strictly statinary white noise, which is not an i.i.d. sequence.

Imagine $Y \tilde (-2 -1 1 2; 1/4 1/4 1/4 1/4)$, $\epsilon_t \tilde iid (-1 1; 1/2 1/2)$. $X_t = Y \epsilon_t$. We calculate $\gamma_X$. It is equal to $Cov(X_h, X_0) = E(X_h X_0) = E(\epsilon_h \epsilon_0 Y^2) = 0$.
\end{excercise}

\begin{observation}
We will ise $(X_t) \tilde IID(0, \sigma^2)$
\end{observation}

\begin{excercise}
Find an example of white noise, which is not even strictly stationary.
\end{excercise}

\begin{example} (random walk)
Assume $(X_t) \tilde IID(\mu, \sigma^2)$, $X_0$ is independent of $(X_t)_{t \geq 1}$ and $Var(X_0) < \infty$.
$X_= = X_0$, $S_n = X_0 + \dots + X_n$ then $(S_n)$ cannot be weakly stationary, unless $\mu = \sigma = 0$.
Clearly if $\mu \neq 0$ then $ES_n = EX_0 + n\mu$ is changing!
If $\mu = 0$ and $\sigma^2 > 0$, then $\gamma_S(h,h) = Cov(S_h, S_h) = Var(S_h) = VarX_0 + n\sigma^2$, which is growing. So we cannot have a statinary random walk.
\end{example}

\begin{observation}
Definition of stationarity (weak and strong) and functions $\gamma_X, \rho_X$ can be extended to one-sided time sreies $(X_t)_ {t\geq 0}$. 
\end{observation}


\begin{example}
Assume $A_1, A_2$ are uncorrelated random variables with expectation $0$ and equal variance $\sigma^2$. Then the time series $X_t = A_1 \cos{\nu t} +A_2 \sin{\nu t}$. $\nu > 0$.
is $w\pi / \nu$ periodic and weakly stationary.
Obviously $EX_t = 0$ and $EX_t^2 < \infty$.
$Cov(X_{t+h}, X_t) = Cov(\dots) = E A_1^2 \cos{\nu t}\cos{t + h} + EA_2^2 \sin{\nu(t + h)}\sin{\nu t} = $
$\sigma^2 \cos(\nu (t + h) - \nu t)$ = $\sigma^2 \nu h$.
There is no $t$ here. So $X_t$ is stationary. 

Just by looking at the sample we cannot know if it is randomly generated. We also do not know if it is stationary (it is $\sin$).
But it is. So it is difficult to analyise in practise.
\end{example}


\begin{excercise}
Assume $A, \theta$ are independant random variables. $Var(A) < \infty$ and $\theta Unif[0,2\pi]$, $\nu > 0$ then the time series
$X_t = A \cos{\nu t + \theta}$ is weakly stationary. 

Solution:
You write the $\cos$ using the trigonometric formula and notice it is the same as in the previous example for 
$A_1 = A\cos{\theta}$ and $A_2 = A\sin{\theta}$.
\end{excercise}

\begin{example} (MA(1))
Assume $(Z_t) \tilde WN(0, \sigma^2), v \in \R$
and $X_t = Z_t + vZ_{t-1}, t \in \Z$.
Process $(X_t)$ is called {\bf moving average of order one} $= MA(1)$ .

Its expectation is clearly $0$. It is also a weakly stationary time series (strongly if $Z_t$ is strongly stationary):
$\gamma_X(h) = Cov(X_{t + h}, X_t) = Cov(Z_{t+h} + vZ_{t+h-1}, Z_t + vZ_{t-1}) = 
=  \sigma^2(1 + v^2)   , h = 0
	v \sigma^2	 , h = +-1
		0	, otherwise
$
This celarly does not depend on $t$, so $(X_t)$ is weakly stationary. We also observe that $\gamma_X(h) = 0$ for all $|h| \geq 2$. We say that $(X_t)$ has {\bf short memory}.

\end{example}

\begin{observation}
Time series theory makes a distinction between:
\begin{itemize}
\item short range dependent models (like MA(1))
\item long range dependent models
\end{itemize}
\end{observation}


\begin{example} (AR(1))
Assume $(Z_t) \tilde WN(0, \sigma^2)$, $\phi \in \R$ and that $(X_t)$ satisfies 
$$X_t = \phi X_{t-1} +Z_t, t \in \Z$$ 
%\formula 
(1)

For different values of $X_0$ we can generate different valies of $(X_t)$ proces. The main question is, if we can (in this setting) find a stationary times series $(X_t)$, which satisfies
% \formula 
(1).

It turns out we cannot always do that. It will depend on $\phi$. $\phi = 1$ gives us a random walk and it is not stationary for nonstationary $(Z_t)$.

If a stationary solution exists it is unique.

Assume $|\phi| < 1$. We can guess the solution by iterating one step backwards. We get:
$X_t = \phi^kX_{t-k} + \phi^{k-1}Z_{t-k} + \dots + Z_t$.
One can guess that the stationary solution when $\phi \rightarrow \infty$ is $X_t = \sum_{k = 0}^\infty \phi^k Z_{t-k}$, which is a random series. 

Does this series converge? Wheat does this mean for random variables? (which convergence?) 

We will show the series is converging a.s.(s.g.) (even absolutly). That will mean $X_t$ is well defined for almost all $\omega-$s. Then it satisfies 
%\formula 
(1) (check directly) and it is weakly stationary. 

$EX_t = E(\sum \phi^k Z_{t-k}) = \sum \phi^k EZ_{t-k} = 0$. 
We would like to put the expectation under the sign (but this is not true neccessarily). 

For $h \geq 0$: 
$\gamma_X(h) = Cov(X_{t +h }, X_t) = E(X_{t+h}X_t) = E(\sum_i \sum_j \phi^i \phi^j Z_{t + h - j} Z_{t + h - i}) = (interchanging)
\sum_i \sum _j \phi^i \phi^j \gamma_Z(h-i+j)$.
(This gives $\sigma^2$ only when $h+j = i$.)
$= \sum_j \phi^j \phi^{h + j}\sigma^2 = \sigma^2 \phi^h \frac{1}{1 - \phi^2}$.

By simmetry we get (for $h \leq 0$):
$\gamma_X(h) = \sigma^2 \frac{\phi^{|h|}}{1 - \phi^2}$.

This is decreasing exponentially fast! We used Lebesgue dominated convergence.


When $\phi = 1$ we cannot find a stationary solution at all:
$X_t = X_{t-1} + Z_t = \dots = X_0 + Z_1 + \dots + Z_t$. Meaning $(X_t)$ is a random walk. 
Assume $Var(X_0) < \infty$. $Var(X_t - X_0) = t \sigma^2$. 
Standard deviation $(X_t - X_0) = \sqrt{t} \sigma$.
(We will show that $|| X || = \sqrt{EX^2}$ is a norm. )
$= \sqrt{E((X_t - X_0) - (EX_t - EX_0))^2} = || X_t - X_0 - (EX_t - EX_0) || \leq || X_t - EX_t || + || X_0 - EX_0|| = s.d(X_t) + s.d(X_0) \in [0, \infty]$.

So when $\sqrt{t}\sigma \rightarrow \infty$ the s.d. also goes towards $\infty$. Therefore $(X_t)$ cannot be weakly stationary (for $\sigma \neq 0$). 
\end{example}



\begin{excercise}
Show that there is no stationary solution of 
%\formula 
(1) if $\phi = -1$.
\end{excercise}

\begin{excercise}
Show that the stationary solution of 
%\formula 
(1) for $|\phi| > 1 $ is of the form 
$X_t = \sum_{k = 1}^\infty (1/\phi)^k Z_{t + k}$.
\end{excercise}

\begin{example}(ARCH(1))
Assume $(S_t)$ are prices of financial assets. Log-returns are $X_t = \log{S_t / S_{t-1}} = \log{1 + \frac{S_t - S_{t-1}}{S_{t-1}}}$. The frac is relative return $X'$. 
$= (approx) X_t'$. So it measures how much money you made (approx).

AR(1) = auto-regressive model of order one:

Typically teople assume that $X_t = \sigma Z:t$, where $Z_t \tilde (iid) N(0,1)$. Black-Scholes model. It is a benchmark model, because it is simple. But it is not correct (realistic).



Engel (1982):
Stylized facts contradict i.i.d. assumption. The normal distribution has small tails. Stock prices go up and down -> clustering.

%\formula 
(2)
$Z_t \tilde iid N(0,1)$ (or more generally $Z_t \tilde IID(0,1)$)
$X_t = \sigma_t Z_t$.
$\sigma_t^2 = \alpha_0 + \alpha_1 X_{t-1}^2$, $\alpha_0, \alpha_1$ konstant.
$\sigma_t^2$ is sometiems called volatility

ARCH = auto-regressive conditionally heteroscadastic (heteroscadastic = variance is changing)

Main question: does 
%\formula 
(2) have a stationary solution? 
For $\alpha_1 \in (0,1)$ one can find a stationary solution with the same trick as before (by iterating backwards).
$\sigma_t^2 = \alpha_0 + \alpha_0 \sum_{j = 1}^\infty \alpha_1 Z_{t-1}^2 \dots \alpha_1Z_{t-j}^2$ 
%\formula 
(3)

This series is converging almost surely (a.s.). 
Series in 
%\formula 
(3) is only made of non-negative terms. 

If we show that the expectaion of this formula is finite than we know that the formula is converging a.s..
$\sigma_t^2 = E(\alpha_0 + \alpha_0 \sum_j \alpha_1^j Z_{t-1}^2 \dots Z_{t-j}^2)$.
Because all terms are non-negative we can exchange $E and \lim$ because of Fubinni (for integrals).
$ = \alpha_0 + \alpha_0 \sum_j \alpha_1^j = \alpha_0 \frac{1}{1-\alpha_1} < \infty$. Therefore $\sigma_t^2 < \infty$ a.s.

The pair $(\sigma_t^2, \sigma_tT_t = X_t)$ is well defined and solves the 
%\formula 
(2).

Observe $\sigma_t \in \sigma\{ Z_{t-q},Z_{t-2}, \dots\}$. So $\sigma_t$ is independent of $Z_t, \dots$.
$E(X_t) = E(E(X_t | \sigma\{\Z_{t-1}, \dots\}))$. We can take $\sigma_t$ out, if it is a constant (and it is, ecause we know the sigma algebra and therefore all values before $t$. In other words $\sigma_t$ is measurable in the sigma algebra). 
$ =E(E(Z_t)\sigma_t)= E(\sigma_t 0) = 0$ .

We can do the same for $E(X_t^2 | \sigma\{\dots\}) = 1 \sigma_t^2$.
$Var(X_t) = EX_t^2 = E(E(\sigma_t^2 Z_t^2 | \sigma\{\dots\})) = E(\sigma_t^2) = \frac{\alpha_0}{1 - \alpha_1}$.

So variance does not change. Only the conditial variance does.

Similarly one can show that $X_t$-s are uncorrelated. So $Cov(X_{t+ h}, X_t) = 0, h \neq 0$. It follows that $(X_t)'$ are white noise. This does not contradict the reality. Prides are not correlated, but are dependant.( Because the sign is unpredictable.)


It was later generalized by Boleslev (1986) into GARCH (generalized ARCH). it is the most often used model, that models real-life data.

In our example $X_t$ is also strictly stationary (it is weakly, because it is white noise):
Since $X_t = \Phi(Z_t, Z_{t-1}, \dots) Z_t'$-s iid. (idea of proof). 
\end{example}

\subsection{Properties of $\gamma$ and $\rho$}

$\gamma_X(h) = Cov(X_{t+h}, X_t), h \in \Z$
$\rho_X(h) = Corr(X_{t+h},X_t) = \gamma_X(h) / \gamma_X(h)$

\begin{lema}
\begin{itemize}
\item $\gamma_X(0) \geq 0$
\item $|\gamma_X(h)| \leq \gamma_X(0)$
\item $\gamma$ is an even function
\item $\gamma$ is a positive semidefinite function i.e. for all $n$ and for all $(a_1, \dots, a_n) \in \R^n$ 
$\sum_{i=1}^\infty \sum_{j=1 }^\infty q_i \gamma(i -j) a_j \geq 0$ 
\end{itemize}
\end{lema}

\proof
\begin{itemize}
\item todo
\item Schwartz-Cauchy inequality
\item todo
\item the sum equals $Var(a_1X_1 + \dots + a_n X_n) =$
$ \sum_{i,j} Cov(a_iX_i,a_jX_j) = \sum a_i a_j \gamma(i-j)$. 
\end{itemize}
\endproof

\begin{observation}
(iv) equals to 
matrix $\Gamma_n = $
$\gamma(0) \gamma(1) \dots \gamma(n-1)$
$\gamma(1)\dots$
$\dots$

is positive semidefinite
\end{observation}


\begin{theorem}
Function $\gamma: \Z \rightarrow \R$ is autocovariant function of some weakly stationary time series 
iff
$\gamma$ is even and positivesemidefinite
\end{theorem}


\proof
(->) proved in lemma
(<-) Because of positive semidefinitnes of $\Gamma_n$ for all $n$ it follows that there exists $N(0, \Gamma_n)$. $\Gamma_n$ can be a covariance matrix. This distributions as a family of f.d.d-s are consistent. Meaning they satisfy the condition of the theorem 
%\ref
 (1).

Theorem (1) now implays that we can find a (gaussian) sequence and has $\Gamma_n$ as its autocovariance function.

\endproof

\begin{observation}
Function $\rho$ has the same properties as function $\gamma$. We also know that $\rho(0) = 1$ always.
\end{observation}

\begin{observation}
It is dificult to test for positive semidifinitnes. Sometimes it is easier to construct a sequence directly.
\end{observation}

\begin{excercise}
Show that the following functions are positive semidefinite.
\begin{itemize}
\item $\gamma(h) = 2$
\item $\gamma(h) =$ 2 if h = 0 1 if h = +-1 0 otherwise
\item $\gamma(h) = a \cos{\nu h}, a, \nu > 0$
\end{itemize}
\begin{itemize}
\item $X_t = X \tilde N(0,2)$
\item is the autocov. of MA model (done before)
\item $\gamma$ is the autocor. of periodic seg. $X_t = \dots$ (done before)
\end{itemize}


\end{excercise}

\begin{excercise}
If $\gamma_1$ and $\gamma_2$ are even and positive semidefinite functions the same holds for the function 
$\gamma = \alpha_1 \gamma_1 + \alpha_2 \gamma_2, \alpha_1, \alpha_2 \geq 0$. 

We sum the two series, multiplied not by $\alpha_1$, but some function of it.

In practise we need to estimate $\gamma_X, \rho_X$ and $\mu = EX_t$. Natural estimators are easy to find:
$\overline{X_n} = 1/n (X_1 + \dots + X_n)$ the arithmetic mean

For iid it converges (law of large numbers). Even a stronger results is the central limit teorem.

We do not have iid, because we have dependance. We can do:
$\hat{\gamma(h)} = $
$1/n \sum_{i = 1}^{n- h}$
$ (X_i - \overline{X_n})(X_{i+h} - \overline{X_n})$.
Obviously $h < n$ must hold. We say $h =$ {\bf  lag}.
We divide by $n$ because such a matrix is again positive definite. But it is not unbiased (we would have to divide by the $n - 1 - $ number of missing terms $= h$). In this case we would not have a positive definite matrix. 

For $h < 0$ we set $\gamma(h) = \gamma(-h)$.

$\hat{\rho(h)} = \hat{\gamma(h)} / \hat{\gamma(0)}$.

\end{excercise}





\subsection{First steps of practical analysis of time series}

;

R libraries:

tseries

TSA

stats

evir

fGarch $->$ not perfectly written ?

mean

sd

read.table

ts.plot $->$ on the same graph (for multivaribale)

plot.ts $->$ on more graphs

acf   plots $\hat{\rho}$

acf also works for squares. Our job is to explain this $->$ we are outside the boundaries for siemens data form R.

arima.sim






























\end{document} 
